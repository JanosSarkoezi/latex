\documentclass[11pt, a4paper]{article}
\usepackage[utf8]{inputenc}
\usepackage[T1]{fontenc}                 % Umlaute und co
\usepackage[ngerman]{babel}              % Deutsche Trennung von Wörtern
\usepackage{graphicx}                    % Für includegraphics
\usepackage{enumitem}                    % Abstände zwischen Listen und Aufzählungen
\usepackage{tikz}                        % Zeichnen von Diagrammen
\usepackage{caption}                     % Beschriftung von Abbildungen
\usepackage{amsmath}                     % Mathematische Gleichungen besser formatieren
\usepackage{pgfplots}                    % Zeichnen von Daten
\usepackage{pgfplotstable}
\usepackage{hyperref}                    % Einbinden von html-Links
\usepackage{csquotes}                    % Markieren von gesprächen

\usetikzlibrary{arrows} 

\tikzstyle{arrow} = [thick,->,>=stealth]
\tikzstyle{box}=[draw, minimum size=2em]

\begin{document}

\begin{figure}[h]
\centering
\begin{tikzpicture}
    \draw [style=help lines] (0, 0) grid +(2, 2)
    \node (s) [box]                                  {A};
    \node (i) [box, node distance=2.2cm, right of=s] {K};
    \node (r) [box, node distance=2.2cm, right of=i] {I};
    \node (d) [box, node distance=2.0cm, below of=i] {T};
    \draw [arrow] (s) -- (i);
    \draw [arrow] (i) -- (r);
    \draw [arrow] (i) -- (d);
\end{tikzpicture}
\caption{Modell SIRD}
\label{fig:sird}
\end{figure}

\end{document}
% vim: tw=92 ts=4 sw=4 et
