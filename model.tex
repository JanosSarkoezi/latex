\documentclass[11pt, a2paper]{article}
\usepackage[utf8]{inputenc}
\usepackage[T1]{fontenc}                 % Umlaute und co
\usepackage[ngerman]{babel}              % Deutsche Trennung von Wörtern
\usepackage{graphicx}                    % Für includegraphics
\usepackage{enumitem}                    % Abstände zwischen Listen und Aufzählungen
\usepackage{tikz}                        % Zeichnen von Diagrammen
\usepackage{caption}                     % Beschriftung von Abbildungen
\usepackage{amsmath}                     % Mathematische Gleichungen besser formatieren
\usepackage{pgfplots}                    % Zeichnen von Daten
\usepackage{pgfplotstable}
\usepackage{hyperref}                    % Einbinden von html-Links
\usepackage{csquotes}                    % Markieren von gesprächen

\usetikzlibrary{arrows}                  % Pfeile zeichnen mit tikZ
\usetikzlibrary{positioning,calc}        % Positionieren von End oder Anfangspunkten
\usetikzlibrary{shapes.geometric}        % Geometrische Figuren

\tikzstyle{arrow} = [thick,->,>=stealth]
\tikzstyle{box}=[draw, minimum size=2em]

\tikzstyle{mybox} = [draw=red, fill=blue!20, very thick,
    rectangle, rounded corners, inner sep=10pt, inner ysep=10pt, minimum size=2em]

\begin{document}

\begin{figure}[h]
\centering
\begin{tikzpicture}
    \node (s) [mybox] {
        \begin{minipage}{4cm}
        To calculate the horizontal position the kinematic differential
        equations are needed:
        For small angles the following approximation can be used:
        \end{minipage}
    };
    \path (s.north east) ++(1.5cm,0.75cm) node[text width=3cm, align=left] (t) {Event\\ bla
    bla bla bla};
    \draw[arrow] (t) -| ($(s.north west)!0.75!(s.north east)$);
\end{tikzpicture}
\caption{Modell SIRD}
\label{fig:1}
\end{figure}

\begin{figure}[h]
\centering
\begin{tikzpicture}
    \node (s) [mybox, text width=5cm, text justified] {
        To calculate the horizontal position the kinematic differential
        equations are needed:
        For small angles the following approximation can be used:
    };
    \path (s.north east) ++(1.5cm,0.75cm) node[text width=3cm, align=left] (t) {Event\\ bla
    bla bla bla};
    \draw[arrow] (t) -| ($(s.north west)!0.75!(s.north east)$);
\end{tikzpicture}
\caption{Modell SIRD}
\label{fig:2}
\end{figure}

\begin{figure}[h]
\centering
\begin{tikzpicture}
    \node (s) [mybox, text width=5cm] {
        To calculate the horizontal position the kinematic differential
        equations are needed:
    };
    \path (s.north east) ++(1.5cm,0.75cm) node[text width=3cm, align=left] (t) {Event\\ bla
    bla bla bla};
    \draw[arrow] (t) -| ($(s.north west)!0.75!(s.north east)$);
    \node (k) [mybox, text width=5cm, above of=s, yshift=-7cm] {
        To calculate the horizontal position the kinematic differential
        equations are needed:
        alll bla bla asdfasdf lasdflasdf 
        alll bla bla asdfasdf lasdflasdf 
        alll bla bla asdfasdf lasdflasdf 
        alll bla bla asdfasdf lasdflasdf 
    };
    \path (k.north east) ++(1.5cm,0.75cm) node[text width=3cm, align=left] (u) {Event\\ bla
    bla bla bla};
    \draw[arrow] (u) -| ($(k.north west)!0.75!(k.north east)$);
    \draw[arrow] (s) -- (k);
\end{tikzpicture}
\caption{Modell SIRD}
\label{fig:3}
\end{figure}


% --------------------------
\tikzstyle{startstop} = [rectangle, rounded corners, minimum width=3cm, minimum height=1cm,text centered, draw=black, fill=red!30]
\tikzstyle{io} = [trapezium, trapezium left angle=70, trapezium right angle=110, minimum width=3cm, minimum height=1cm, text centered, draw=black, fill=blue!30]
\tikzstyle{process} = [rectangle, minimum width=3cm, minimum height=1cm, text centered, text width=3cm, draw=black, fill=orange!30]
\tikzstyle{decision} = [diamond, minimum width=3cm, minimum height=1cm, text centered, draw=black, fill=green!30]
\tikzstyle{arrow} = [thick,->,>=stealth]

\begin{tikzpicture}[node distance=2cm]

\node (start) [startstop] {Start};
\node (in1) [io, below of=start] {Input};
\node (pro1) [process, below of=in1] {Process 1};
\node (dec1) [decision, below of=pro1, yshift=-0.5cm] {Decision 1};
\node (pro2a) [process, below of=dec1, yshift=-0.0cm] {Process 2a text text text text text text text text text text};
\node (pro2b) [process, right of=dec1, xshift=2cm] {Process 2b};
\node (out1) [io, below of=pro2a] {Output};
\node (stop) [startstop, below of=out1] {Stop};

\draw [arrow] (start) -- (in1);
\draw [arrow] (in1) -- (pro1);
\draw [arrow] (pro1) -- (dec1);
\draw [arrow] (dec1) -- node[anchor=east] {yes} (pro2a);
\draw [arrow] (dec1) -- node[anchor=south] {no} (pro2b);
\draw [arrow] (pro2b) |- (pro1);
\draw [arrow] (pro2a) -- (out1);
\draw [arrow] (out1) -- (stop);

\end{tikzpicture}

\end{document}
% vim: tw=92 ts=4 sw=4 et
