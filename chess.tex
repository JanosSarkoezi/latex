\documentclass[10pt]{article}
\usepackage{doc}
\usepackage{xskak}

\begin{document}
\setfigfontfamily{merida}
\setboardfontfamily{merida}

\section{Theoritical section}
Chess has passed throught a lenthty course of a development and today it is
continuing its evolution so swiftly, that yesterday's evolutions look
increasingly obsolete. The rise in the importance of the competitive factor
is the most marked tendency of modern chess.

The deciding of the 1997 world championship in a rapid-play game is the best
demonstration of this thesis. Victory in chess ist certainly the undisputed
aim, but this factor should not prevail over the search for the truth,
however difficult it may be.\\

\textbf{As Alekhine wrote back in the 1920s, in chess it is important not 
wath, but how.}\\

With the apperance of computers, the technical level and the level of 
opening knowledge ist constantly rising.

Now the battle between two opponents passes through several critical points.\\

\textbf{In many games the hierarchy of strategic factors, determining the evaluation
of a position, varies, and plans and ideas are transforme.
It is this that constitutes dynamism in chess.}\\

The method on which I have worked, and which I offer here, enables the
dynamic evolution of strategic elements in a chess game to be foreseen, for
them to be analysed, and, in the end, for this process to be controlled.

This short paragraph comprises in concentrated form the move search algotitm
in chess.

Thus to foresee \textbf{the modification of the hierarchy of strategic
factors} is nothing other than to be able to define critical positions.
\textbf{I suggest analysing critical positions on the basis of their
static state,} without taking account of dynamic factors.\\

\textbf{This aim is served by the proposed static balance.
Candidate moves are chosen in accordance with the static balance.\\
Here we should perhaps dwell on the concepts of 'static' and 'dynamic'
factors.\\
By 'static' are implied factors that have an enduring effect.\\
Whereas dynamic factors are associated with a change in the state of a
position,with the energy of a breakthrouth, with the coming into contact
with the opposing army.\\
With the passage of time their role diminishes and reduces ho noutht.}\\

Imagine the following situation: one of the warring sides ist shut up
castle, surrounded by the enemy. A whole series of factor influence
action of the opponents. Thus for the example abstence of food or drinking
water may force the castle defenders to engage in an open battle. Otherwise
it may be better to strengthen the walls in the expectation of winter, when
the enemy will be forced to untertake a dubious storm, in order to not lose
a significant part of their army.\\

For lady chess players I could suggest another comparsion: between classical
clothes and footwear and other corresponding to the latest fashion, sometime
rather extravagant. The latter are more costly, and demand immediate wearing,
sice soon it will be hard to find any use of them.\\

\textbf{In for one of the player static balance is negative, he must without
hesitation employ dynamic means, and be ready to go in for extreme measures.}\\

\begin{center}
    \textbf{\Large A brief resume}
\end{center}
The move search algotitm:\\

\textbf{1) Find a critical position (a turning point in the play, a moment when
there is a possible change in the hierarchy of strategic elements).}\\

\textbf{2) Draw up the static balance of this postion , allowing it to be dediced
who in the following phase should use a static and who a dynamic means.}\\

\textbf{3) Consider the candidate moves and chose a specific move.}\\

Between critical positions there are technical phases.\\

\textbf{In my view, the separation of a game into opening, middlegame and endgame
has no practical use.}\\

\textbf{To some extent it is even harmful, since already at very early stage a game
often pass through several critical positions.}
\newpage
\textbf{\Large Definition of a critical position}\\

I suggest three criteria for the existence of a critical postion.\\

\textbf{1)} A position in which a decision has to be taken regarding \textbf{a
\underline{possible} exchange.} If the exchange is forced there is no change compared
with the previous critical position.\\

\textbf{2)} A position in which a decision has to be taken regarding \textbf{a
\underline{possible} change in the pawn formation.} Especially of the central pawns.\\

\textbf{3) The end of a series of forced moves.} Here one should not draw a parallel
between  force moves and the moves relating to a combination.\\

To sense that a position is critical is already a great sucess.

\begin{center}
\twocolumn
\textsf{NN -- NN}\\
\textit{Example game}
\end{center}
\newchessgame
\mainline{1. e4 e5 2. Nf3 Nc6 3. Bb5 a6 4. Ba4 Nf6 5. O-O Be7 6. Re1 b5 7. Bb3 d6 8. c3 O-O 9. h3 Na5 10. Bc2 c5 11. d4 Qc7 12. Nbd2 Nc6 13. d5 Na5 14. b3 c4 15. b4 Nb7 16. a4}

\begin{center}
    \chessboard[smallboard, showmover, label=false]
\end{center}
\end{document}
