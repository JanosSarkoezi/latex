\documentclass[10pt]{article}
\usepackage{doc}
\usepackage{texmate}

\begin{document}
\setfigfontfamily{merida}
\setboardfontfamily{merida}

\section{Theoritical section}
Chess has passed throught a lenthty course of a development and today it is
continuing its evolution so swiftly, that yesterday's evolutions look
increasingly obsolete. The rise in the importance of the competitive factor
is the most marked tendency of modern chess.

The deciding of the 1997 world championship in a rapid-play game is the best
demonstration of this thesis. Victory in chess ist certainly the undisputed
aim, but this factor should not prevail over the search for the truth,
however difficult it may be.

As Alekhine wrote back in the 1920s, in chess it is important not wath, but
how.

With the apperance of computers, the technical level and the level of 
opening knowledge ist constantly rising.

Now the battle between two opponents passes through several critical points.

In many games the hierarchy of strategic factors, determining the evaluation
of a position, varies, and plans and ideas are transforme.

It is this that constitutes dynamism in chess.

The method on which I have worked, and which I offer here, enables the
dynamic evolution of strategic elements in a chess game to be foreseen, for
them to be analysed, and, in the end, for this process to be controlled.

This short paragraph comprises in concentrated form the move search algotitm
in chess.

Thus to foresee the modification of the hierarchy of strategic factors is
nothing other than to be able to define critical positions. I suggest
analysing critical positions on the basis of their static state, without
taking account of dynamic factors.

This aim is served by the proposed static balance.

Candidate moves are chosen in accordance with the static balance.

Here we should perhaps dwell on the concepts of 'static' and 'dynamic'
factors.

By 'static' are implied factors that have an enduring effect.

Whereas dynamic factors are associated with a change in the state of a
position,with the energy of a breakthrouth, with the coming into contact
with the opposing army.

With the passage of time their role diminishes and reduces ho noutht.

...
\newgame
\twocolumn
\renewcommand\afterno{~}
\renewcommand\iiopent{\par}
\renewcommand\iicloset{\par}
\renewcommand\takes{x}
\renewcommand\beforeb{\the\move.~\dots\ }

\begin{center}
\textsf{NN -- NN}

\textit{Example game}
\end{center}
\noindent|1. e4 e5 2. Nf3 Nc6 3. Bb5 a6 4. Ba4 Nf6 5. 0-0 Be7 6. Re1 b5 7. Bb3 d6 8. c3 0-0 9. h3 Na5 10. Bc2 c5 11. d4 Qc7 12. Nbd2 Nc6 13. d5 Na5 14. b3 c4 15. b4 Nb7 16. a4|

\begin{center}
\showboard
\end{center}
\end{document}
