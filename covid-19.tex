\documentclass[11pt, a4paper]{article}
\usepackage[utf8]{inputenc}
\usepackage[T1]{fontenc}                 % Umlaute und co
\usepackage[ngerman]{babel}              % Deutsche Trennung von Wörtern
\usepackage{graphicx}                    % Für includegraphics
\usepackage{enumitem}                    % Abstände zwischen Listen und Aufzählungen
\usepackage{tikz}                        % Zeichnen von Diagrammen
\usepackage{caption}                     % Beschriftung von Abbildungen
\usepackage{amsmath}                     % Mathematische Gleichungen besser formatieren
\usepackage{pgfplots}                    % Zeichnen von Daten
\usepackage{pgfplotstable}
\usepackage{hyperref}                    % Einbinden von html-Links

\usetikzlibrary{arrows} 

\tikzstyle{arrow} = [thick,->,>=stealth]
\tikzstyle{box}=[draw, minimum size=2em]

\begin{document}

\begin{titlepage}
    \title{Die Jagt der COVID-19}
    \author{Janos Sarközi}
    \maketitle
\end{titlepage}

\section{Einführung}

Die Ausbreitung eines Viruses kann durch die Modellierung eines dynamischen System erreicht
werden. Dafür wird die Befölkerung in vier Gruppen unterteilt.

\begin{itemize}[itemsep=0pt]
    \item Anfällig
    \item Krank
    \item Immun
    \item Tot
\end{itemize}

Zwischen diesen Personen wird durch die Krankheit ein Austausch statt finden. Dies kann
in ein Diagramm zusammengefasst werden. Dabei kürzen wir die vier Zustände Anfällig = A,
Krank = K, Immun = I und Tot = T ab.

\begin{figure}[h]
\centering
\begin{tikzpicture}
    \node (s) [box]                                  {A};
    \node (i) [box, node distance=2.2cm, right of=s] {K};
    \node (r) [box, node distance=2.2cm, right of=i] {I};
    \node (d) [box, node distance=2.0cm, below of=i] {T};
    \draw [arrow] (s) -- (i);
    \draw [arrow] (i) -- (r);
    \draw [arrow] (i) -- (d);
\end{tikzpicture}
\caption{Zustandsänderung der Befölkerung}
\label{fig:sird}
\end{figure}

In Abbildung \ref{fig:sird} ist folgendes zu sehen. Aus anfälligen Personen können durch
den Virus Kranke personen werden. Die kranken Personen können gesund werden und eine
Immunität daruch erlangen oder Sie sterben. Es stellt sich noch dir Frage, ob immune
Personen doch nicht die Immunität haben wie gedacht und wieder anfällig werden können?
Hier ein Bild dazu.

\begin{figure}[h]
\centering
\begin{tikzpicture}
    \node (s) [box]                                  {A};
    \node (i) [box, node distance=2.2cm, right of=s] {K};
    \node (r) [box, node distance=2.2cm, right of=i] {I};
    \node (d) [box, node distance=1.5cm, below of=i] {T};
    \coordinate[above of=r] (d1);
    \coordinate[above of=s] (d2);
    \draw [arrow] (s) -- (i);
    \draw [arrow] (i) -- (r);
    \draw [arrow] (i) -- (d);
    \draw [arrow] (r) -- (d1) -- (d2) -- (s);
\end{tikzpicture}
\caption{Rückfall aus der Immunität}
\label{fig:noImmune}
\end{figure}

Einfachheits halber, sollen die immune Personen immun bleiben. In der Abbildung
\ref{fig:sird} soll auch ein einfaches Modell betrachtet werden. Natürlich gitb es auch
noch kompliziertere Modelle, aber sehen wir mal an, wie weit wir mit diesem Modell aus der
Abbildung \ref{fig:sird} kommen.

\section{Etwas Mathematik}
Erweitert man die Abbildung 1 wie folgt:
\begin{figure}[h]
\centering
\begin{tikzpicture}
    \node (s) [box]                                  {A};
    \node (i) [box, node distance=2.2cm, right of=s] {K};
    \node (r) [box, node distance=2.2cm, right of=i] {I};
    \node (d) [box, node distance=2.0cm, below of=i] {T};
    \draw [arrow] (s) -- (i) node[midway, above=2pt, fill=none] {$\beta$AK};
    \draw [arrow] (i) -- (r) node[midway, above=2pt, fill=none] {$\gamma$K};
    \draw [arrow] (i) -- (d) node[midway, right=2pt, fill=none] {$\delta$K};
\end{tikzpicture}
\caption{Dynamisches Modell}
\label{fig:dynModel}
\end{figure}

Dann kann ein Differentialgleichungssystem aufgeschrieben werden. Es beschreibt die
Änderung der Zustände der Personen.

\begin{equation}
    \begin{aligned}
        \frac{dA}{dt} &= -\beta AK                      \\[5pt]
        \frac{dK}{dt} &= \beta AK - \gamma K - \delta K \\[5pt]
        \frac{dI}{dt} &= \gamma K                       \\[5pt]
        \frac{dT}{dt} &= \delta K                       \\[10pt]
    \end{aligned}
    \label{eq:sird}
\end{equation}

Diese Gleichungen können leicht aus der Abbildung \ref{fig:dynModel} aufgeschrieben werden.

\begin{itemize}[itemsep=0pt]
    \item Das Minuszeichen in der ersten Gleichung erklärt sich durch die Abwanderung der
        Anfälligen zu den Kranken.
    \item Die Zuwanderung der Anfälligen zu den Kranen, ist der erste Teil der zweiten Gleichung.
        Der rest der zweiten Gleichung beschreibt den Sachchverhalt, dass aus den Kranken
        immune oder gestorbene werden.
    \item Die dritte und vierte Gleichung übernimmt nur die Kranken, sodass sie entweder
        gesund oder gestorben sind.
\end{itemize}

\newpage
\section{Lösen der Gleichungen}

Für das Lösen des Gleichungssystems kramt man sich aus dem Internet etwas raus. Ich habe
mich für  Jupyther Notebook entschieden. Das Ergebnis sieht dann so aus:

\begin{figure}[h]
\centering
\begin{tikzpicture}
    \begin{axis} [xlabel=Zeit, ylabel=relative Häufigkeit, legend cell align={left}]
        \addplot [line width=1pt, color=blue, no markers]
          table [x=x, y=S, col sep=comma] {data/sird.csv};
        \addlegendentry{Anfällig}
        \addplot [line width=1pt, color=red, no markers]
          table [x=x, y=I, col sep=comma] {data/sird.csv};
        \addlegendentry{Krank}
        \addplot [line width=1pt, color=green, no markers]
          table [x=x, y=R, col sep=comma] {data/sird.csv};
        \addlegendentry{Immun}
        \addplot [line width=1pt, no markers]
          table [x=x, y=D, col sep=comma] {data/sird.csv};
        \addlegendentry{Tot}
    \end{axis}
\end{tikzpicture}
\caption{Dynamisches Modell}
\label{plot:sird}
\end{figure}

Berichtet wird meinst nur über die Kranken und Tote. Damit fokusieren wir und nur noch auf
diese. Ein Bild sagt mehr aus tausend Worte.

\begin{figure}[h]
\centering
\begin{tikzpicture}
    \begin{axis} [xlabel=Zeit, ylabel=relative Häufigkeit, legend cell align={left}]
        \addplot [line width=1pt, color=red, no markers]
          table [x=x, y=I, col sep=comma] {data/sird.csv};
        \addlegendentry{Krank}
        \addplot [line width=1pt, no markers]
          table [x=x, y=D, col sep=comma] {data/sird.csv};
        \addlegendentry{Tot}
    \end{axis}
\end{tikzpicture}
\caption{Betrachtug von Infizierten und Gestorbenen}
\label{plot:id}
\end{figure}

Damit wird klar was passieren kann nach diesem Modell. Die Infizierten nehmen explosionsartig
zu um eine maximale Grüße zu erreichen und danach etwas langsamer abzulklingen. Die Steigung der
Kurven variieren sicherlich von Land zu Land. Eine schöne Darstellung für die einzelnen Staaten
ist under der Adresse \url{https://epidemic-simulator.now.sh/#/} zu finden. Die genannte Seite
benutzt ebenfalls die Gleichungen \ref{eq:sird} für die Darstellung der möglichen Entwicklung
der pandemie in den einzelnen Ländern.

\end{document}
