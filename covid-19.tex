\documentclass[11pt, a4paper]{article}
\usepackage[utf8]{inputenc}
\usepackage[T1]{fontenc}                 % Umlaute und co
\usepackage[ngerman]{babel}              % Deutsche Trennung von Wörtern
\usepackage{graphicx}                    % Für includegraphics
\usepackage{enumitem}                    % Abstände zwischen Listen und Aufzählungen
\usepackage{tikz}                        % Zeichnen von Diagrammen
\usepackage{caption}                     % Beschriftung von Abbildungen
\usepackage{amsmath}                     % Mathematische Gleichungen besser formatieren
\usepackage{pgfplots}                    % Zeichnen von Daten
\usepackage{pgfplotstable}
\usepackage{hyperref}                    % Einbinden von html-Links
\usepackage{csquotes}                    % Markieren von gesprächen

\usetikzlibrary{arrows} 

\tikzstyle{arrow} = [thick,->,>=stealth]
\tikzstyle{box}=[draw, minimum size=2em]

\begin{document}

\begin{titlepage}
    \title{Die Jagt nach der COVID-19}
    \author{Janos Sarközi}
    \maketitle
\end{titlepage}

\section{Ich brauche Waffen!}

Hmmm ... welche Waffen könnte ich gebrauchen. Ja irgendwas, womit ich weit schießen kann. Ich nehme
mir ein einfaches Modell gengen COVID-19. Was kann der Virus? Anfällige Personen kran machen und
manche kranke töten. Also schreibe ich mir die Gruppen auf, wo mein Jagtziel sich rumtreibt.

\begin{itemize}[itemsep=0pt]
    \item Anfällig
    \item Krank
    \item Immun
    \item Tot
\end{itemize}

Ich gehe in ein Waffenladen und frage einen Verkäufer: \textquote{Welches Modell nehme ich?}
Der Verkäufer: \textquote{Was wollen Sie den Jagen?} Darauf hin ich \textquote{Einen Virus}.
Der Verkäufer ramt aus seinem Schubladen ein Plan raus und legt ihn auf den Tisch.
\textquote{Nehmen Sie das Modell SIRD. Das ist die englische Abkürzug. Hier kennt man die
besser under AKIT}

\begin{figure}[h]
\centering
\begin{tikzpicture}
    \node (s) [box]                                  {A};
    \node (i) [box, node distance=2.2cm, right of=s] {K};
    \node (r) [box, node distance=2.2cm, right of=i] {I};
    \node (d) [box, node distance=2.0cm, below of=i] {T};
    \draw [arrow] (s) -- (i);
    \draw [arrow] (i) -- (r);
    \draw [arrow] (i) -- (d);
\end{tikzpicture}
\caption{Modell SIRD}
\label{fig:sird}
\end{figure}

Ich sehe mir das Bild \ref{fig:sird} an. Da erzählt der Verkäufer mir die Funktionsweise meiner
Waffe. \textquote{Mit A sind die anfällignen Personen gemeint, K sind die Kranken, I die
Immunen und T die Toten.} \textquote{Aha} sage ich. Dann spricht er weiter. \textquote{Die
Pfeile deuten auf die Veränderung der Personen hin. Zum Beispiel wird eine anfälligen Person
durch eine kranke Person angestecht und wird damit selbst zur kranken Person. Das is der
Pfeil zwischen A und K. Wo der Virus dann verlohren hatt, sind die Personen, die Immunität
erlangt haben. Dies Zeigt der Pfeil zwischen K und I an. Allerdigs gibt es auch Verlierer,
dass sind die mit dem T.}  \textquote{Aha, dann bedeutet der Pfeil von K nach T, dass der
Virus paar Leute getötet hat?} \textquote{Ja}, sagt der Verkäufer.

\newpage
Der Verkäufer sieht das ich kauffreudig bin und bietet mir ein etwas besseres Modell an.
\textquote{Schauen Sie sich mal das Modell SIRID} und liegt mir ein andere Beschreibung vor die
Nase. Oh denke ich, das sieht etwas komplizierter aus, aber ich lasse den Verkäufer ausreden.
\begin{figure}[h]
\centering
\begin{tikzpicture}
    \node (s) [box]                                  {A};
    \node (i) [box, node distance=2.2cm, right of=s] {K};
    \node (r) [box, node distance=2.2cm, right of=i] {I};
    \node (d) [box, node distance=1.5cm, below of=i] {T};
    \coordinate[above of=r] (d1);
    \coordinate[above of=s] (d2);
    \draw [arrow] (s) -- (i);
    \draw [arrow] (i) -- (r);
    \draw [arrow] (i) -- (d);
    \draw [arrow] (r) -- (d1) -- (d2) -- (s);
\end{tikzpicture}
\caption{Rückfall aus der Immunität}
\label{fig:noImmune}
\end{figure}

\textquote{Der Virus ist hier stärker, denn wie Sie sehen geht hier ein Pfeil von I nach A.
Das heißt, die Immunen sind garnicht immun geworden, sonder können wieder zu den Anfälligen
gehöhren und dann sind sie wieder ansteckbar.} \textquote{Ne ne}, sage ich \textquote{mir
reicht die einfachere Variante. Geben Sie mit das Modell AKIT}

\section{Die Waffe in meiner Hand}
Der Händler holt eine Waffe von hinten und gibt es mir in die Hand. Sie ist ganz neu. Da
entdecke ich eine Gravur auf der Seite der Waffe. \textquote{Das sieht ja genau so aus, wie das
Bild, dass Sie mir gezeigt haben.} Dabei Zeige ich auf die Gravur der Waffe. \textquote{Moment,
da ist noch mehr zu sehen.}

\begin{figure}[h]
\centering
\begin{tikzpicture}
    \node (s) [box]                                  {A};
    \node (i) [box, node distance=2.2cm, right of=s] {K};
    \node (r) [box, node distance=2.2cm, right of=i] {I};
    \node (d) [box, node distance=2.0cm, below of=i] {T};
    \draw [arrow] (s) -- (i) node[midway, above=2pt, fill=none] {$\beta$AK};
    \draw [arrow] (i) -- (r) node[midway, above=2pt, fill=none] {$\gamma$K};
    \draw [arrow] (i) -- (d) node[midway, right=2pt, fill=none] {$\delta$K};
\end{tikzpicture}
\caption{Die Gravur}
\label{fig:gravur}
\end{figure}

\textquote{Was sind diese komischen Zeichen über den Pfeilen?} Frage ich den Händler.
\textquote{Das haben sich Mathematiker ausgedacht.} Sagt er. \textquote{Aber als guter
Händler habe ich mir eine Notiz in meinem Büchlein gemacht, für Kunden die sich dafür
Interessieren. Warten sie ein Moment ich hole das Buch.} Kuze Zeit später ist er mit einem
braunen lederbezogenen Buch wider zurück. Sieht sehr gepflegt aus. Öffent das Buch und dreht
es zu mir, sodas ich seine Notizen lesen kann.

\textquote{Das hier sind Differentialgleichungen.} Er Zeigt mit seinem Finger auf die erste
Gleichung. \begin{equation}
    \begin{aligned}
        \frac{dA}{dt} &= -\beta AK                      \\[5pt]
        \frac{dK}{dt} &= \beta AK - \gamma K - \delta K \\[5pt]
        \frac{dI}{dt} &= \gamma K                       \\[5pt]
        \frac{dT}{dt} &= \delta K                       \\[10pt]
    \end{aligned}
    \label{eq:sird}
\end{equation}
\textquote{Die erste Gleichung beschreibt die Veränderung der anfällignen Personen. Das sehen
Sie auf der linken Seite. Auf der rechten Seite entdecken sie die Zeichen, die Sie auf der
Gravur oberhalb des Pfeiles zwischen A und K gesehen hatten.} Dabei zeigt er nun auf die
Gravur auf der Waffe in meiner Hand.

\begin{figure}[h]
\centering
\begin{tikzpicture}
    \node (s) [box]                                  {A};
    \node (i) [box, node distance=2.2cm, right of=s] {K};
    \draw [arrow] (s) -- (i) node[midway, above=2pt, fill=none] {$\beta$AK};
\end{tikzpicture}
\caption{ein Teil der Gravur}
\label{fig:teilGravur}
\end{figure}

\textquote{Nur war da kein Minuszeichen. Das kommt daher, weil die anfälligen Personen nach
der Zeit immer wehniger werden. Als Eselsbrücke, kann man sich merken, wenn ein Pfeil von
einem Kasten wegzeigt, wird das was über dem Pfeil steht mit einem Minuszeichen versehen.}
Er nimmt seine Hand zurück und legt seine Finger die zwiete Zeile der Gleichungen.
\textquote{Wenn Sie sich erinnern hat der Kasten mit dem K, drei Pfeile um sich.  Eins geht
in Richtung K und zwei gehen weg von K. Schon wird klar was die zeite Gleichung beschreibt.}
Ich schaue auf die Waffe in meiner Hand.

\begin{figure}[h]
\centering
\begin{tikzpicture}
    \node       (i) [box, node distance=2.2cm] {K};
    \coordinate [node distance=2.2cm, left of=i]  (s);
    \coordinate [node distance=2.2cm, right of=i] (r);
    \coordinate [node distance=1.5cm, below of=i] (d);
    \draw [arrow] (s) -- (i) node[midway, above=2pt, fill=none] {$\beta$AK};
    \draw [arrow] (i) -- (r) node[midway, above=2pt, fill=none] {$\gamma$K};
    \draw [arrow] (i) -- (d) node[midway, right=2pt, fill=none] {$\delta$K};
\end{tikzpicture}
\caption{Die Gravur}
\label{fig:gravur2}
\end{figure}

\textquote{Ja}, sage ich \textquote{die Personen die anfällig waren sind nun krank geworden
und von den Kranken weden welche immun oder sterben ... leider.} \textquote{Na sehen Sie, so
einfach ist es. Die dritte und vierte Gleichung muss ich garnicht mehr erklären.} Ich fange
an innerlich zu strahlen und merke, dass ich lächle. \textquote{Ja, die Waffe will ich haben!}

\section{Benzutung der Waffe}

\textquote{Sehr schön.} Sagt der Händler. \textquote{Eine Schießübung müssen Sie noch
machen. Worauf wollen sie schießen?} Frage er mich. \textquote{Zeigen Sie erstmal
alle Zeile.} \textquote{Na gut, kommen Sie mit raus auf das Feld, da können wir mit der
Schießübung anfangen. Aber erstmal müssen Sie noch die Waffe richgit einstellen. Schauen
Sie auf die Gravur Ihrer Waffe. Dort sehen Sie neben den bekannten Buchstaben AKIT auch
solche wie $\beta$, $\gamma$ und $\delta$.} 

\begin{figure}[h]
\centering
\begin{tikzpicture}
    \node (s) [box]                                  {A};
    \node (i) [box, node distance=2.2cm, right of=s] {K};
    \node (r) [box, node distance=2.2cm, right of=i] {I};
    \node (d) [box, node distance=1.5cm, below of=i] {T};
    \draw [arrow] (s) -- (i) node[midway, above=2pt, fill=none] {$\beta$AK};
    \draw [arrow] (i) -- (r) node[midway, above=2pt, fill=none] {$\gamma$K};
    \draw [arrow] (i) -- (d) node[midway, right=2pt, fill=none] {$\delta$K};
\end{tikzpicture}
\caption{Die Gravur}
\label{fig:grauvur3}
\end{figure}

\textquote{Das sind drei Schräubchen auf Ihrer Waffe, die Sie einstellen müssen um das Zeil
genau zu treffen.} Er nimm die Waffe aus meiner Hand und Zeigt mir die drei stellen.
\textquote{Hier, hier und hier.} Zeitg dabei einmal eine Schraube über das Zielfernrohr,
dann am Vorderschaft und schließlich in der nähe des Abzuges. Er holt sich noch sein
Büchlein raus und zeigt mir ein Beispiel für eine Einstellung.

\begin{figure}[h]
\centering
\begin{tikzpicture}[scale=0.9]
    \begin{axis} [xlabel=Zeit, ylabel=relative Häufigkeit, legend cell align={left}]
        \addplot [line width=1pt, color=blue, no markers]
          table [x=x, y=S, col sep=comma] {data/sird.csv};
        \addlegendentry{Anfällig}
        \addplot [line width=1pt, color=red, no markers]
          table [x=x, y=I, col sep=comma] {data/sird.csv};
        \addlegendentry{Krank}
        \addplot [line width=1pt, color=green, no markers]
          table [x=x, y=R, col sep=comma] {data/sird.csv};
        \addlegendentry{Immun}
        \addplot [line width=1pt, no markers]
          table [x=x, y=D, col sep=comma] {data/sird.csv};
        \addlegendentry{Tot}
    \end{axis}
\end{tikzpicture}
\caption{Eine Einstellmöglichkeit der Waffe}
\label{plot:sird}
\end{figure}

\newpage
Wow, denke ich und Sage dann: \textquote{Was für ein durcheinander. Was ist das denn: Zeit und relative
Häufigkeit?} \textquote{Ihr Ziel, der Virus, ist beweglich. Springt mit der Zeit von einer
Person zu einer anderen Person. Daher die Zeit. Relative Häufigkeit besagt, zum Beispiel
bei den kranken nur die Zahl, die entsteht, wenn Sie die Anzahl der kranken Personen durch
die Anzahl alle Personen nehmen. Das ist Ihr rote Kurve. Die blauen sind die anfälligen,
die grünen die immnue und die schwarzen sind die toten Personen.} Da kommt mir eine Frage
auf \textquote{Wie weit muss ich an den Schräubchen drehen, damit ich das richtige Ziel
treffe?} \textquote{Da habe ich eine schöne Internetseite für Sie. Haben Sie was zum
Schreiben?} \textquote{Ja, kann los gehen.} Er sagt mit die Adresse
\url{https://epidemic-simulator.now.sh/#/}. Dann fügte er noch hinzu, dass auf der Seite
nur zwei der vier Ergebnisse der Einstellungen zu sehen sind und meinte noch, dass für kenner
diese beiden Kurven ausreichen. \textquote{Hier sehen Sie mal, so oder ähnlich sehen die
Bilder im Internet aus.}

\begin{figure}[h]
\centering
\begin{tikzpicture}
    \begin{axis} [xlabel=Zeit, ylabel=relative Häufigkeit, legend cell align={left}]
        \addplot [line width=1pt, color=red, no markers]
          table [x=x, y=I, col sep=comma] {data/sird.csv};
        \addlegendentry{Krank}
        \addplot [line width=1pt, no markers]
          table [x=x, y=D, col sep=comma] {data/sird.csv};
        \addlegendentry{Tot}
    \end{axis}
\end{tikzpicture}
\caption{Betrachtug von Infizierten und Gestorbenen}
\label{plot:id}
\end{figure}

\textquote{Am besten suchen sie sich Germany raus. Damit können Sie dann hier in Deutschland
auf die Jagt gehen.} Ich bedankte mich für die gute Beratung. Nach dem Bezahlen und mit
voller Freude ging ich mit meiner neuen Waffe nach Hause.

\end{document}
