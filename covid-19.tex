\documentclass[11pt, a4paper]{article}
\usepackage[utf8]{inputenc}
\usepackage[T1]{fontenc}                 % Umlaute und co
\usepackage[ngerman]{babel}              % Deutsche Trennung von Wörtern
\usepackage{graphicx}
\usepackage{enumitem}                    % Abstände zwischen Listen und Aufzählungen
\usepackage{wrapfig}
\usepackage{tikz}                        % Zeichnen von Diagrammen
\usepackage{caption}                     % Beschriftung von Abbildungen
\usepackage{amsmath}                     % Mathematische Gleichungen besser formatieren

\usetikzlibrary{arrows} 

\tikzstyle{arrow} = [thick,->,>=stealth]
\tikzstyle{box}=[draw, minimum size=2em]

\begin{document}

\begin{titlepage}
    \title{Die gedanken eines Mathematikers über die COVID-19}
    \author{Janos Sarközi}
    \maketitle
\end{titlepage}

\section{Einführung}

Die Ausbreitung eines Viruses kann durch die Modellierung eines dynamischen System erreicht
werden. Dafür wird die Befölkerung in vier Gruppen unterteilt.

\begin{itemize}[itemsep=0pt]
    \item Anfällig
    \item Krank
    \item Immun
    \item Tot
\end{itemize}

Zwischen diesen Personen wird durch die Krankheit ein Austausch statt finden. Dies kann
in ein Diagramm zusammengefasst werden. Dabei kürzen wir die vier Zustände Anfällig = A,
Krank = K, Immun = I und Tot = T ab.

\begin{figure}[h]
\centering
\begin{tikzpicture}
    \node (s) [box]                                  {A};
    \node (i) [box, node distance=2.2cm, right of=s] {K};
    \node (r) [box, node distance=2.2cm, right of=i] {I};
    \node (d) [box, node distance=2.0cm, below of=i] {T};
    \draw [arrow] (s) -- (i);
    \draw [arrow] (i) -- (r);
    \draw [arrow] (i) -- (d);
\end{tikzpicture}
\caption{Zustandsänderung der Befölkerung}
\end{figure}

In Abbildung 1 ist folgendes zu sehen. Aus anfälligen Personen können durch den Virus Kranke personen werden. Die kranken Personen können gesund werden und eine Immunität daruch
erlangen oder Sie sterben. Es stellt sich noch dir Frage, ob immune Personen doch nicht die
Immunität haben wie gedacht und wieder anfällig werden können? Einfachheits halber, sollen
die immune Personen immun bleiben. In der Abbildung 1 soll auch ein einfaches Modell
betrachtet werden. Es gibt natürlich auch kompliziertere Modelle, aber sehen wir mal an,
wie weit wir mit diesem Modell kommen.

\newpage
\section{Etwas Mathematik}
Erweitert man die Abbildung 1 wie folgt:
\begin{figure}[h]
\centering
\begin{tikzpicture}
    \node (s) [box]                                  {A};
    \node (i) [box, node distance=2.2cm, right of=s] {K};
    \node (r) [box, node distance=2.2cm, right of=i] {I};
    \node (d) [box, node distance=2.0cm, below of=i] {T};
    \draw [arrow] (s) -- (i) node[midway, above=2pt, fill=none] {$\beta$AK};
    \draw [arrow] (i) -- (r) node[midway, above=2pt, fill=none] {$\gamma$K};
    \draw [arrow] (i) -- (d) node[midway, right=2pt, fill=none] {$\delta$K};
\end{tikzpicture}
\label{dynModel}
\caption{Dynamisches Modell}
\end{figure}

Dann kann ein Differentialgleichungssystem aufgeschrieben werden. Es beschreibt die
Änderung der Zustände der Personen.

\begin{equation}
    \begin{aligned}
        \frac{dA}{dt} &= -\beta AK                      \\[5pt]
        \frac{dK}{dt} &= \beta AK - \gamma K - \delta K \\[5pt]
        \frac{dI}{dt} &= \gamma K                       \\[5pt]
        \frac{dT}{dt} &= \delta K                       \\
    \end{aligned}
\end{equation}

Die erste Gleichung beschreibt die Änderung der anfälligen Personen. Das Minuszeichen auf
der rechten Seite der Gleichung besagt, dass die Anzahl der anfälligen Personen mit der
Zeit weniger wird. Die zweite Gleichung ist ewas komplizierter. Aber wenn  die Abbildung
\ref{dynModel} betrachtet wird, so wird sie verständlich. Der Pfeil geht von A nach K, also
werden genauso viele Personen in K eintreffen wie aus A rausgenommen wurden. Weiterhin
wandern aus K Personen nach I und T. Schließlich die Personen, die aus K gewandert sind,
kommen in I und T an. Dies sagen die letzten beiden Gleichungen aus.

Für das Lösen des Gleichungssystems kramt man sich aus dem Internet etwas raus. Ich habe
mich für  Jupyther Notebook entschieden.
\end{document}
