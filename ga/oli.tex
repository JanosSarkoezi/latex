\documentclass[11pt, a4paper, fleqn]{report}
\usepackage{amsmath}
\usepackage{tikz}
\usepackage[ngerman]{babel}
\usepackage[utf8]{inputenc}
\usetikzlibrary {arrows.meta, angles, quotes}

% Numerierung der Formel innerhalb einer Sektion
\numberwithin{equation}{section}

% Abkürzung für mathbf
\def\*#1{\mathbf{#1}}

\title{Geometrische Algebra \\
für Oliver Hut \\
von Janos Sarközi}

\begin{document}
\maketitle
\section{Historisches}
Die Idee der geometrischen Algebra war schon 1844 Hermann Grassmann in seinem
Buch über die Ausdehnungslehre entwickelt worden. Parallel dazu hatte sich
Willima Rowan Hamilton mit der Drehung von Objekten im Raum beschäftigt.
Initial erkannte Hamilton, dass die Drehung von Objekten in der Ebene mit den
komplexen Zahlen erreicht werden kann. Aus seiner Arbeit von Drehungen von
Objekten im Raum entstanden die Quaternionen. Seine berühmte Formel
\[i^2=j^2=k^2=ijk=-1\]
ist in Dublin auf der Boomer Bridge auf einer Tafel verewigt worden. Dort
hatte er den Geistesblitz während eines Spazierganges mit seiner Frau.
Grassmann und Hamilton wussten möglicher Weise von der Arbeit des anderen,
aber erkannten nicht die Ähnlichkeiten. Nur ein junger Mathematiker namens
William Kingdon Clifford erkannte dies. Da er schon mit 33 gestorben war,
er konnte sich nicht in der Welt der Mathematiker etablieren, wurde seine
Arbeit keine große Aufmerksamkeit gewidmet.

Über hundert Jahre vergingen, bis sich jemand wieder mit diesen Themen
befasste. Er nahm die Arbeiten von Grassman, Hamilton und Clifford auf und
führte die geometrische Algebra auf neue gefilde. Diese Person ist David
Hestenes.

\section{Motivation und Ideen}
Die Seele der ganzen geometrische Algebra lässt sich in die
Gleichung
\[ab = a\cdot b + a\wedge b\]
schreiben. Heutige Bezeichnung von $a$ und $b$ sind Vektoren. Dabei bedeutet
$a\cdot b$ das Skalarprodukt oder das innere Produkt und $a\wedge b$ das
äußere Produkt. Die Formel besagt also, dass das geometrische Produkt zweier
Vektoren $a$ und $b$ sich aus dem inneren und äußeren Produkt berechnet
werden kann. Es wird kein neues Zeichen für das geometrische Produkt
eingeführt.

Verlangt wurde von diesem geometrischen Produkt, dass sie assoziativ und 
distributiv ist
\begin{align*}
    a(bc) &= (ab)c \\
    a(b+c) &= ab + ac
\end{align*}
\newpage
Wenden wir uns nun den einzelnen Komponenten des geometrischen Produktes zu.
\section{Das innere Produkt}
Das innere Produkt ist uns allen durch die Formel bekannt
\[a\cdot b = |a||b|\cos\alpha \]
Wobei $\alpha$ der Winkel zwischen den Vektoren $a$ und $b$ ist. Visuell sieht
das so aus.
\begin{figure}[h]
\centering
\begin{tikzpicture}[>=Stealth]
    \coordinate (A) at (0,0);
    \coordinate (B) at (3,0);
    \coordinate (D) at (1,1.5);

    \draw[->, thick] (A) -- (B);
    \draw[->, thick] (A) -- (D);
    \node[shift={(-0.4,-0.3)}] (v) at (D) {$v$};
    \node[shift={(-0.5,-0.2)}] (u) at (B) {$u$};
    \pic["$\alpha$", ->, draw=black, angle radius=0.8cm, angle eccentricity=0.6] {angle=B--A--D};

\end{tikzpicture}
\caption{Das Skalarprodukt zwischen den Vektoren $a$ und $b$}
\label{fig:innerProduct}
\end{figure}
\end{document}
