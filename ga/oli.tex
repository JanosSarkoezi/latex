\documentclass[fleqn]{scrartcl}
\usepackage{amsmath}
\usepackage{amsfonts}
\usepackage{mathtools}
\usepackage{tikz}
\usetikzlibrary{calc}
\usepackage[ngerman]{babel}
\usepackage[utf8]{inputenc}
\usetikzlibrary {arrows.meta, angles, quotes}

% Numerierung der Formel innerhalb einer Sektion
\numberwithin{equation}{section}

% Abkürzung für mathbf
\def\*#1{\mathbf{#1}}

\title{Geometrische Algebra \\
für Oliver Hut}
\author{Janos Sarközi}

\begin{document}
\maketitle
\newpage
\section{Historisches}
Die Idee der geometrischen Algebra war schon 1844 Hermann Grassmann in seinem
Buch über die Ausdehnungslehre entwickelt worden. Parallel dazu hatte sich
Willima Rowan Hamilton mit der Drehung von Objekten im Raum beschäftigt.
Initial erkannte Hamilton, dass die Drehung von Objekten in der Ebene mit den
komplexen Zahlen erreicht werden kann. Aus seiner Arbeit von Drehungen von
Objekten im Raum entstanden die Quaternionen. Seine berühmte Formel
\[i^2=j^2=k^2=ijk=-1\]
ist in Dublin auf der Boomer Bridge auf einer Tafel verewigt worden. Dort
hatte er den Geistesblitz während eines Spazierganges mit seiner Frau.
Grassmann und Hamilton wussten möglicher Weise von der Arbeit des anderen,
aber erkannten nicht die Ähnlichkeiten. Nur ein junger Mathematiker namens
William Kingdon Clifford erkannte dies. Da er schon mit 33 gestorben war,
er konnte sich nicht in der Welt der Mathematiker etablieren, wurde seine
Arbeit keine große Aufmerksamkeit gewidmet.

Über hundert Jahre vergingen, bis sich jemand wieder mit diesen Themen
befasste. Er nahm die Arbeiten von Grassman, Hamilton und Clifford auf und
führte die geometrische Algebra auf neue Gefilde. Diese Person ist David
Hestenes.

\section{Motivation und Ideen}
Die Seele der ganzen geometrische Algebra lässt sich in der
Gleichung
\[ab = a\cdot b + a\wedge b\]
erkennen. Heutige Bezeichnung von $a$ und $b$ sind Vektoren. Dabei bedeutet
$a\cdot b$ das Skalarprodukt oder das innere Produkt und $a\wedge b$ das
äußere Produkt. Die Formel besagt also, dass das geometrische Produkt zweier
Vektoren $a$ und $b$ sich aus dem inneren und äußeren Produkt berechnet
werden kann. Es wird kein neues Zeichen für das geometrische Produkt
eingeführt.

Verlangt wurde von diesem geometrischen Produkt, dass sie assoziativ und 
distributiv ist
\begin{align*}
    a(bc) &= (ab)c \\
    a(b+c) &= ab + ac
\end{align*}
Wenden wir uns nun den einzelnen Komponenten des geometrischen Produktes zu.
\subsection{Das Gummiband und der Stab}
Wir werden viel mit Parallelogramme in der geometrischen Algebra arbeiten.
Daher eine Idee für die Fläche erhaltende Modifikation eines
Parallelogramms. Betrachten wir die untere Seite des Parallelogramms als fest
verankert, die linke und rechte Seite aus einem Gummiband und die obere Seite
als ein Stab (der selben Länge, wie die unterer Seite).
\shorthandoff{"}
\begin{center}
\begin{minipage}{\linewidth}
\centering
\begin{tikzpicture}[>=Stealth]
    \draw (-0.5, 0.0) -- (3.5, 0.0);
    \draw (-1.0, 1.5) -- (4.5, 1.5);
    \coordinate (O) at ( 0.0, 0.0);
    \coordinate (A) at ( 3.0, 0.0);
    \coordinate (B) at (-0.5, 1.5);
    \draw [draw=black, fill=green, opacity=0.5]
    (O) -- (A) -- ($(A)+(B)$) -- (B) -- cycle;

    \coordinate (B) at ( 1.0, 1.5);
    \draw [draw=black, fill=green, opacity=0.2]
    (O) -- (A) -- ($(A)+(B)$) -- (B) -- cycle;
\end{tikzpicture}
\captionof{figure}{Der Stab nach der Linksverschiebung}
\label{fig:gummiUndStab1}
\end{minipage}
\end{center}
\shorthandon{"}
Auf der Abbildung \ref{fig:gummiUndStab1} wurde der Stab nach links
verschoben. Die alte Position des Parallelogramms ist noch durchsichtig
angedeutet. Schieben wir den Stab noch weiter nach links, so könnte das so
aussehen.
\shorthandoff{"}
\begin{center}
\begin{minipage}{\linewidth}
\centering
\begin{tikzpicture}[>=Stealth]
    \draw (-0.5, 0.0) -- (3.5, 0.0);
    \draw (-3.0, 1.5) -- (4.5, 1.5);
    \coordinate (O) at ( 0.0, 0.0);
    \coordinate (A) at ( 3.0, 0.0);
    \coordinate (B) at (-2.5, 1.5);
    \draw [draw=black, fill=green, opacity=0.5]
    (O) -- (A) -- ($(A)+(B)$) -- (B) -- cycle;

    \coordinate (B) at ( 1.0, 1.5);
    \draw [draw=black, fill=green, opacity=0.2]
    (O) -- (A) -- ($(A)+(B)$) -- (B) -- cycle;
\end{tikzpicture}
\captionof{figure}{Der Stab nach veitere Linksverschiebung}
\label{fig:gummiUndStab2}
\end{minipage}
\end{center}
\shorthandon{"}
Verändern wir die Initialkonfiguration des Parallelogramms so, das jetzt
die linke Seite fest verankert ist, die rechte Seite ein Stab wird und die
obere bzw. untere Seite aus einem Gummiband besteht. Nun verschieben wir die
rechte Seite schräg nach unten und erhalten.
\shorthandoff{"}
\begin{center}
\begin{minipage}{\linewidth}
\centering
\begin{tikzpicture}[>=Stealth]
    \coordinate (O) at ( 0.0, 0.0);
    \coordinate (A) at ( 3.0, 0.0);
    \coordinate (B) at ( 1.0, 1.5);
    \draw ($-0.2*(B)$) -- ($1.2*(B)$);
    \draw ($(A)+1.2*(B)$) -- ($(A)-1.8*(B)$);
    \draw [draw=black, fill=green, opacity=0.5]
    (O) -- (B) -- ($(A) - 0.5*(B)$) -- ($(A) - 1.5*(B)$) -- cycle;

    \coordinate (B) at ( 1.0, 1.5);
    \draw [draw=black, fill=green, opacity=0.2]
    (O) -- (A) -- ($(A)+(B)$) -- (B) -- cycle;
\end{tikzpicture}
\captionof{figure}{Neue Konfiguration. Der Stab nach links unten verschoben}
\label{fig:gummiUndStab3}
\end{minipage}
\end{center}
\shorthandon{"}
All diese Parallelogramme haben den selben Flächeninhalt. Später führen wie
noch eine Orientierung ein, die auch bei allen Parallelogrammen erhalten
bleibt.

Ab jetzt können wir den Gummi-Stab Trick auf Parallelogramme anwenden und
wissen was gemeint ist.
\newpage
\subsection{Das innere Produkt}
Das innere Produkt ist uns allen durch die Formel bekannt
\[a\cdot b = |a||b|\cos\alpha \]
Wobei $\alpha$ der Winkel zwischen den Vektoren $a$ und $b$ ist. Visuell sieht
das so aus.
\shorthandoff{"}
\begin{center}
\begin{minipage}{\linewidth}
\centering
\begin{tikzpicture}[>=Stealth]
    \coordinate (O) at (0,0);
    \coordinate (A) at (3,0);
    \coordinate (B) at (1,1.5);

    \draw[->, thick] (O) -- (A);
    \draw[->, thick] (O) -- (B);
    \node[shift={(-0.5,-0.3)}] (v) at (B) {$b$};
    \node[shift={(-0.5,-0.3)}] (u) at (A) {$a$};
    \pic["$\alpha$", -, draw=black, angle radius=0.8cm, angle eccentricity=0.6] {angle=A--O--B};
\end{tikzpicture}
\captionof{figure}{Das Skalarprodukt zwischen den Vektoren $a$ und $b$}
\label{fig:innerProduct}
\end{minipage}
\end{center}
\shorthandon{"}

\subsection{Das äußere Produkt}
Das äußere Produkt der Vektoren $a$ und $b$ soll die orientierte Fläche des
Parallelogramms werden, das durch diese Vektoren aufgespannt wird. Hier ein
Bild
\shorthandoff{"}
\begin{center}
\begin{minipage}{\linewidth}
\centering
\begin{tikzpicture}[>=Stealth]
    \coordinate (O) at (0,0);
    \coordinate (A) at (3,0);
    \coordinate (B) at (1,1.5);
    \draw [draw=black, fill=green, opacity=0.2]
    (O) -- (A) -- ($(A)+(B)$) -- (B) -- cycle;

    \draw[->, thick] (O) -- (A);
    \draw[->, thick] (O) -- (B);
    \node[shift={(-0.5,-0.3)}] at (B) {$b$};
    \node[shift={(-0.5,-0.3)}] at (A) {$a$};
    \pic ["$\alpha$", -, draw=black, angle radius=0.8cm, angle eccentricity=0.6] {angle=A--O--B};
    \draw[->, semithick] ([shift=(0:0.3)]1.8,0.8) arc (0:270:0.3);
\end{tikzpicture}
\captionof{figure}{Das äußere Produkt zwischen den Vektoren $a$ und $b$}
\label{fig:outherProduct}
\end{minipage}
\end{center}
\shorthandon{"}

Die Orientierung der Fläche wird durch die Reihenfolge der Vektoren $a$ und
$b$ in dem äußeren Produkt $a\wedge b$ bestimmt und wird mit einem kleinen
Kreissegment mit Pfeil gekennzeichnet. Wird das äußere Produkt von $b$ mit $a$
gebildet, so soll eine andere Orientierung enstehen. Da es nur zwei
Orientierungen gibt wird die Beziehung
\[a\wedge b = - b\wedge a\]
zwischen den äußeren Produkten gelten. Die Fläche des Parallelogramms lässt
sich mit der Gleichung
\[|a\wedge b| = |a||b||\sin\alpha|\]
beschreiben.

Hier enstehen die Ähnlichkeiten zwischen einem Vektor und einer orientierten
Fläche. Beide haben eine Größe und eine Orientierung. Die Größe nennt sich bei
dem Vektor die Länge und bei dem äußeren Produkt die Fläche. Die Orientierung
bei der Fläche hat auch den Namen links bzw. rechts orientiert oder mit dem
Uhrzeigersinn bzw. gegen dem Uhrzeigersinn.

Eine andere Möglichkeit der Darstellung der Orientierung ist die Vektoren, wie
bei der Addition an einander zu reihen.
\shorthandoff{"}
\begin{center}
\begin{minipage}{\linewidth}
\centering
\begin{tikzpicture}[>=Stealth]
    \coordinate (O) at (0,0);
    \coordinate (A) at (3,0);
    \coordinate (B) at (1,1.5);
    \coordinate (C) at ($(A) + (B)$);
    \draw [draw=black, fill=green, opacity=0.2]
    (O) -- (A) -- ($(A)+(B)$) -- (B) -- cycle;

    \draw[->, thick] (O) -- (A);
    \draw[->, thick] (A) -- ($(A) + (B)$);
    \node[shift={(-0.5,-0.3)}] at (A) {$a$};
    \node[shift={( 0.2,-0.3)}] at ($(A) + (B)$) {$b$};
    \node at ($(O)!0.5!(C)$) {$a\wedge b$};
\end{tikzpicture}
\captionof{figure}{Andere Darstellung des äußeren Produktes}
\label{fig:outherProduct2}
\end{minipage}
\end{center}
\shorthandon{"}

Paar interessante Eigenschaften des äußeren Produktes. Zuerst eine einfache.
Liegt der Vektor $b$ in der Richtung von $a$ so ist das äußere Produkt Null.
Wenn also $b=\lambda a$ ist, so soll
\[b\wedge a = (\lambda a)\wedge a= 0\]
sein. Das leuchtet ein, da dann die Fläche des Parallelogramms Null ist. Das
folgt aus der genannten Gleichung über die Orientierung der Fläche
\[a\wedge b = - b\wedge a \textrm{ ist äquivalent zu } a\wedge b + b\wedge a = 0\]
Setzen wir $b = a$ ein, so kommt
\[2(a\wedge a)=0\]
raus und damit
\[a\wedge a = 0\]

Da das äußere Produkt als ein Produkt bezeichnet wird, soll auch die
Distributivität gelten.
\[a\wedge (b + c) = a\wedge b + a\wedge c\]
In der Ebene ist diese Gleichung leicht einzusehen. Betrachte die folgende
Abbildung
\shorthandoff{"}
\begin{center}
\begin{minipage}{\linewidth}
\centering
\begin{tikzpicture}[>=Stealth]
    \coordinate (O) at ( 0.0, 0.0);
    \coordinate (A) at (-0.5, 3.0);
    \coordinate (B) at ( 4.0, 0.0);
    \coordinate (C) at ( 3.0, 2.0);
    \draw [draw=black, fill=green, opacity=0.3]
    (O) -- (B) -- ($(A) + (B)$) -- (A) -- cycle;
    \draw [draw=black, fill=green, opacity=0.3]
    (B) -- ($(B) + (C)$) -- ($(A) + (B) + (C)$) -- ($(A) + (B)$) -- cycle;
    \draw [draw=black, fill=yellow, opacity=0.3]
    (O) -- ($(B) + (C)$) -- ($(A) + (B) + (C)$) -- (A) -- cycle;

    \draw[->, thick] (O) -- (A);
    \draw[->, thick] (A) -- ($(A) + (B)$);
    \draw[->, thick] (B) -- ($(A) + (B)$);
    \draw[->, thick] ($(A) + (B)$) -- ($(A) + (B) + (C)$);
    \draw[->, thick] (A) -- ($(A) + (B) + (C)$);

    \node[shift={(-0.2,-0.6)}] (a) at (A) {$a$};
    \node[shift={(-0.5,-0.3)}] (b) at ($(A) + (B)$) {$b$};
    \node[shift={(-0.5,-0.6)}] (c) at ($(A) + (B) + (C)$) {$c$};
\end{tikzpicture}
\captionof{figure}{Distributivität des äußeren Pruduktes}
\label{fig:distributivity}
\end{minipage}
\end{center}
\shorthandon{"}
Schaeun wir die Parallelogramme der orientierten Flächen $a\wedge b$ und
$a\wedge c$ an. Sie sind Grün eingefärbt. Das Parallelogramm der orientierten
Fläche $a\wedge (b + c)$ ist Gelb. Aus dem Bild ist leicht zu erkennen, dass
die Summe der grünen Flächen gleich dem der gelben Fläche ist. Die
Orientierungen der drei Flächen sind alle im Uhrzeigersinn.
\newpage
Nun können wir uns auf die Gleichheit von orientierten Fläche stürzen. Denken
wir uns ein Parallelogramm so, dass der Vektor $a$ horizontal liegt und der
Vektor $b$ schräg nach rechst oben zeigt. Wenden wir den Gummi-Stab Trick so
an, dass der Vektor $a$ fest verankert ist und der Vektor $b$ zu einem
Gummiband wird, und verschieben den oberen Stab nach links. Dann wird der
Vektor $b$ zu einem Vektor $b'=b+\lambda a$. Setzen wir das ein und rechnen
aus.
\[a\wedge b' = a\wedge (b + \lambda a) = a \wedge b + a\wedge (\lambda a)=
a\wedge b\]
\shorthandoff{"}
\begin{center}
\begin{minipage}{\linewidth}
\centering
\begin{tikzpicture}[>=Stealth]
    \coordinate (O) at (0,0);
    \coordinate (A) at (3,0);
    \coordinate (B) at (1,1.5);
    \coordinate (C) at ($(A)+(B)$);
    \draw [draw=black, fill=green, opacity=0.2]
    (O) -- (A) -- (C) -- (B) -- cycle;

    \draw[->, thick] (O) -- (A);
    \draw[->, thick] (A) -- (C);
    \node[shift={( 0.2,-0.3)}] at (C) {$b$};
    \node[shift={(-0.5,-0.3)}] at (A) {$a$};
    \node at ($(O)!0.5!(C)$) {$a\wedge b$};
    \begin{scope}[shift={(5.5,0)}]
        \coordinate (O) at (0,0);
        \coordinate (A) at (3,0);
        \coordinate (B) at (-0.3,1.5);
        \coordinate (C) at ($(A)+(B)$);
        \draw [draw=black, fill=green, opacity=0.2]
        (O) -- (A) -- (C) -- (B) -- cycle;

        \draw[->, thick] (O) -- (A);
        \draw[->, thick] (A) -- (C);
        \node[shift={( 0.4,-0.3)}] at (C) {$b'$};
        \node[shift={(-0.5,-0.3)}] at (A) {$a$};
        \node at ($(O)!0.5!(C)$) {$a\wedge b'$};
    \end{scope}
\end{tikzpicture}
\captionof{figure}{Der Gummi-Stab Trick einmal angewendet}
\label{fig:shiftb}
\end{minipage}
\end{center}
\shorthandon{"}

Wende wir den Gummi-Stab Trick auf den so entstandenes Parallelogramm an.
Jetzt wird $b'$ verankert und $a$ zu einem Gummiband. Nach dem Gummi-Stab
Trick könnte das so aussehen.
\shorthandoff{"}
\begin{center}
\begin{minipage}{\linewidth}
\centering
\begin{tikzpicture}[>=Stealth]
    \coordinate (O) at (0,0);
    \coordinate (A) at (3,0);
    \coordinate (B) at (-0.3,1.5);
    \coordinate (C) at ($(A)+(B)$);
    \draw [draw=black, fill=green, opacity=0.2]
    (O) -- (A) -- (C) -- (B) -- cycle;

    \draw[->, thick] (O) -- (A);
    \draw[->, thick] (A) -- (C);
    \node[shift={( 0.4,-0.3)}] at (C) {$b'$};
    \node[shift={(-0.5,-0.3)}] at (A) {$a$};
    \node at ($(O)!0.5!(C)$) {$a\wedge b'$};
    \begin{scope}[shift={(5.5,0)}]
        \coordinate (O) at (0,0);
        \coordinate (A) at (2.9,0.5);
        \coordinate (B) at (-0.3,1.5);
        \coordinate (C) at ($(A)+(B)$);
        \draw [draw=black, fill=green, opacity=0.2]
        (O) -- (A) -- (C) -- (B) -- cycle;

        \draw[->, thick] (O) -- (A);
        \draw[->, thick] (A) -- (C);
        \node[shift={( 0.4,-0.3)}] at (C) {$b'$};
        \node[shift={(-0.5,-0.3)}] at (A) {$a'$};
        \node at ($(O)!0.5!(C)$) {$a'\wedge b'$};
    \end{scope}
\end{tikzpicture}
\captionof{figure}{Der Gummi-Stab Trick nochmal angewendet}
\label{fig:shiftshift}
\end{minipage}
\end{center}
\shorthandon{"}

Die Flächen sind gleich, die Orientierung ist gleich ... Ahaaa ... Ich kann
also beliebige Flächen betrachten, solange nur die Orientierung und Größe der
Flächen identisch ist ändert sich das äußere Produkt nicht. Somit kann das
äußere Produkt weiter verallgemeinert werden. Es kann als ein Kreis, eine
Ellipse oder ein wabbeliges Ding dargestellt werden.

Wie steht es um die Assoziativität des äußeren Produktes? Wenn wir das
wünschen, dann verlassen wir die zweite Dimension und treten in die dritte
Dimension ein. Wir verlangen also
\[a\wedge(b\wedge c) =(a\wedge b)\wedge c.\]
Was für ein Ding ist das vor mir? Das ist ein orientiertes Volumen. Die
Orientierung bestimmen die Vektoren $a, b$ und $c$. Betrachten wir die
Analogien zwischen dem orientierten Parallelogramm und dem orientierten
Parallelepiped. Das orientierte Parallelogramm hat orientierte Längen
(Vektoren) am Rand und das orientierte Parallelepiped orientierte
Parallelogramme.
\subsection{Eingenschaften des geometrischen Produktes}
Durch die eben beschriebenen Eingenschaften des inneren und äußeren Produktes,
können wir paar Eingenschaften des geometrischen Produktes herleiten. Aus der
Gleichung
\[ab = a\cdot b + a\wedge b\]
sind folgende Punkte ersichtlich
\begin{itemize}
\item Wenn $a \cdot b = 0$, dann $ab = a \wedge b$.
  \begin{itemize}
  \item Das heißt, wenn $a \perp b$, dann $ab = a\wedge b$.
  \item Weiterhin gilt auch wenn $a \perp b$, dann $ab = a \wedge b = - b
      \wedge a = -ba$
  \end{itemize}
\item Wenn $a\wedge b = 0$, dann $ab = a \cdot b$.
  \begin{itemize}
  \item Das heißt, wenn $a \parallel b$, dann $ab = a \cdot b$.
  \end{itemize}
\end{itemize}

Aber halt ... was haben wir gemacht? Eine Zahl $a\cdot b$ mit einer orientierten
Fläche $a\wedge b$ addiert? Haben wir nicht Äpfel mit Birnen verglichen bzw.
addiert? Wenn wir ehrlich sind, dann machen wir das gleiche bei den komplexen
Zahlen. Denn ist $z$ eine komplexe Zahl, so schreiben wir ganz mutig
\[z=x+yi\]
wobei $x$ und $y$ reelle Zahlen sind und $i$ die berühmte imaginäre Zahl ist.
Diese imaginäre Zahl multiplizieren wir sogar mit einer reellen Zahl und
addieren das ganze zu einer reellen Zahl?! Da mischen wir auch Äpfel mit
Birnen und das geht sogar recht gut! Also lass uns eine reelle Zahl $a\cdot b$
mit einer orientierten Fläche $a\wedge b$ addieren und schauen was rauskommt.

Ist $a \parallel b$, dann ist, wie wir oben fest gestellt haben, $ab = a
\cdot b$. Setzen wir $b=a$ ein, dann kommt
\[aa=a\cdot a = |a|^2\]
raus. Teilen wir beide Seiten mit $|a|^2$, so erhalten wir
\[\frac{aa}{|a|^2}=1.\]
Wenn wir jetzt noch zulassen, das wir die positive reelle Zahl $|a|^2$ auf
eines der Vektoren verteilen dürfen, dann erhalten wir
\[a\frac{a}{|a|^2}=1\]
Durch ein scharfes hinsehen erkennen wir das $\frac{a}{|a|^2}$ zu $a$ invers
ist. Denn das geometrische Produkt von $a$ mit $\frac{a}{|a|^2}$ ist eins. Wow
... wir dürfen mit einem Vektor teilen?! Das ist neu!

Wie meine ist das mit einem Vektor teilen. Jeder Vektor (ungleich Null) hat
eine Inverse bzgl. des geometrischen Produktes. Damit können wir zum Beispiel
die Gleichung
\[ab=c\]
umschreiben in
\[a=cb^{-1}=c\frac{b}{|b|^2}=c\frac{b}{b\cdot b}\]
und wenn wir cheaten, können wir mit $b$ kürzen und erhalten
\[a=\frac{c}{b}\]
Diese Gleichung ist aber mit vorsicht zu genießen, da wir aus ihr nicht sehen,
ob wir die links oder die rechts Inverse von $b$ genommen haben!
\section{Geometrische Algebra in der Ebene}
Wir haben jede menge Zeugs gesammelt bzw. motiviert. Sehen wir uns an, was auf
der (euklidischen) Ebene mit der geometrischen Algebra alles anstellen kann.
Dazu nehmen wir unsere alt bekannten Basisvektoren $e_1$ und $e_2$ aus der
Kiste und rechnen damit etwas mit dem (neuen) geometrischen Produkt paar
Sachen aus. Dabei soll immer stillschweigend Orthonormalität dieser Vektoren
angenommen werden. Was ist das geometrische Produkt dieser Vektoren?
\[e_1e_1=e_1\cdot e_1=1\textrm{ und auch }e_2e_2=1\]
Das $e_1e_1 = e_1\cdot e_1$ ist, folgt aus der Parallelität des Vektors $e_1$
mit sich selbst. Weiter folgt aus dieser Gleichung, dass $e_1$ und $e_2$ zu
sich selbst invers sind. OK, was ist $e_1$ mit $e_2$ verheiratet? Erstmal die
Größe der Fläche des Rechtecks (Parallelogramms).
\[|e_1e_2|=|e_1\wedge e_2|=|e_1||e_2||\sin(\pi/2)|=1\textrm{ und auch
}|e_2e_1|=1\]
Da die beiden Vektoren $e_1$ und $e_2$ orthogonal sind ergibt sich aus der
Kette von Gleichungen
\[e_1e_2=e_1\wedge e_2 = -e_2\wedge e_1 = -e_2e_1\]
Damit nicht immer ein Kreis mit einem Pfeil für die Orientierung gezeichnet
werden muss, ist auch die Anreihung der Vektoren $e_1$ und $e_2$ an einander
eine Möglichkeit.
\shorthandoff{"}
\begin{center}
\begin{minipage}{\linewidth}
\centering
\begin{tikzpicture}[>=Stealth]
    \coordinate (O)  at (0.0, 0.0);
    \coordinate (e1) at (1.5, 0.0);
    \coordinate (e2) at (0.0, 1.5);
    \draw [draw=black, fill=green, opacity=0.2]
    (O) -- (e1) -- ($(e1)+(e2)$) -- (e2) -- cycle;

    \draw[->, thick] (O) -- (e1);
    \draw[->, thick] (e1) -- ($(e1) + (e2)$);
    \draw[->, thick] ($(e1) + (e2)$) -- (e2);
    \draw[->, thick] (e2) -- (O);
    \node[label=below:{$e_1$}] at (0.75, 0.00) {};
    \node[label=left:{$-e_2$}] at (0.00, 0.75) {};
    \node[label=right:{$e_2$}] at (1.50, 0.75) {};
    \node[label=above:{positiv orientiert}] at (0.75, 1.8) {$-e_1$};
    \node[label=center:{$e_1\wedge e_2$}] at (0.75, 0.75) {};
    \begin{scope}[shift={(5.5,0.0)}]
        \coordinate (O)  at (0.0, 0.0);
        \coordinate (e1) at (1.5, 0.0);
        \coordinate (e2) at (0.0, 1.5);
        \draw [draw=black, fill=green, opacity=0.2]
        (O) -- (e1) -- ($(e1)+(e2)$) -- (e2) -- cycle;

        \draw[->, thick] (O) -- (e2);
        \draw[->, thick] ($(e1) + (e2)$) -- (e1);
        \draw[->, thick] (e2) -- ($(e1) + (e2)$);
        \draw[->, thick] (e1) -- (O);
        \node[label=below:{$-e_1$}] at (0.75, 0.00) {};
        \node[label=left:{$e_2$}]   at (0.00, 0.75) {};
        \node[label=right:{$-e_2$}] at (1.50, 0.75) {};
        \node[label=above:{negativ orientiert}] at (0.75, 1.8) {$e_1$};
    \node[label=center:{$e_2\wedge e_1$}] at (0.75, 0.75) {};
    \end{scope}
\end{tikzpicture}
\captionof{figure}{Orientierungen der Bivektoren $e_1\wedge e_2$ und
    $e_2\wedge e_1$ in der Ebene}
\label{fig:orientationPlane}
\end{minipage}
\end{center}
\shorthandon{"}
\newpage
Das ganze kann noch einfacher im Bild dargestelt werden, wenn die negatvien
Vektoren weggelassen werden.
\shorthandoff{"}
\begin{center}
\begin{minipage}{\linewidth}
\centering
\begin{tikzpicture}[>=Stealth]
    \coordinate (O)  at (0.0, 0.0);
    \coordinate (e1) at (1.5, 0.0);
    \coordinate (e2) at (0.0, 1.5);
    \draw [draw=black, fill=green, opacity=0.2]
    (O) -- (e1) -- ($(e1)+(e2)$) -- (e2) -- cycle;

    \draw[->, thick] (O) -- (e1);
    \draw[->, thick] (e1) -- ($(e1) + (e2)$);
    \node[label=below:{$e_1$}] at (0.75, 0.00) {};
    \node[label=right:{$e_2$}] at (1.50, 0.75) {};
    \node[label=above:{positiv orientiert}] at (0.75, 1.8) {};
    \node[label=center:{$e_1\wedge e_2$}] at (0.75, 0.75) {};
    \begin{scope}[shift={(5.5,0.0)}]
        \coordinate (O)  at (0.0, 0.0);
        \coordinate (e1) at (1.5, 0.0);
        \coordinate (e2) at (0.0, 1.5);
        \draw [draw=black, fill=green, opacity=0.2]
        (O) -- (e1) -- ($(e1)+(e2)$) -- (e2) -- cycle;

        \draw[->, thick] (O) -- (e2);
        \draw[->, thick] (e2) -- ($(e1) + (e2)$);
        \node[label=left:{$e_2$}]   at (0.00, 0.75) {};
        \node[label=above:{negativ orientiert}] at (0.75, 1.8) {$e_1$};
    \node[label=center:{$e_2\wedge e_1$}] at (0.75, 0.75) {};
    \end{scope}
\end{tikzpicture}
\captionof{figure}{Orientierungen der Bivektoren $e_1\wedge e_2$ und
    $e_2\wedge e_1$ in der Ebene}
\label{fig:orientationPlane2}
\end{minipage}
\end{center}
\shorthandon{"}
Jetzt wird es spannend. Was ist das Quadrat des Bivektors $e_1e_2$ bzgl. des
geometrischen Produktes?
\[(e_1e_2)^2\stackrel{1}=(e_1e_2)(e_1e_2)\stackrel{2}=e_1(e_2e_1)e_2\stackrel{3}=e_1(-e_1e_2)e_2\stackrel{4}=-(e_1e_1)(e_2e_2)\stackrel{5}=-1\]
\begin{enumerate}
    \item Ausschreiben des Quadrats bzgl des geometrischen Produktes.
    \item Assoziativität des geometrischen Produktes.
    \item $e_2e_1$ ist, wie wir weiter oben berechnet haben $-e_1e_2$
    \item Erneut Assoziativität und vorziehen der Zahl $-1$.
    \item Das geometrische Produkt von $e_1e_1$ bzw. $e_2e_2$ ist $1$.
\end{enumerate}
Ein neues Aha Moment! Nochmal das Ergebnis ohne (störende) Zwischenrechnung
\[(e_1e_2)^2=-1\]
Das Quadrat von $e_1e_2$ ist $-1$?! Das ist doch wie bei der komplexen Zahl
$i$. Nur hier ist alles reell. Wir haben also unsere Einheit der Ebene
gefunden, mit der zusätzlichen Eigenschaft, das ihr Quadrat gleich minus eins
ist.

Um die Einheit $e_1e_2 = e_1\wedge e_2$ nutzen zu können vereinbaren wir noch
den Bereich von $\alpha$. Wenn $-\pi\leq\alpha\leq\pi$ ist, dann kann jedes
äußere Produkt als
\[a\wedge b= |a||b|\sin\alpha \hspace{1ex} e_1\wedge e_2\]
geschrieben werden. Wollen wir Ähnlichkeiten zu den komplexen Zahlen erkennen,
so definieren wir ein fettes $\*i$ als
\[\*i:= e_1\wedge e_2\]
und erhalten so
\[a\wedge b= |a||b|\sin\alpha\*i\]
Erinnern wir uns an die motivierende Gleichung der geometrischen Algebra, so
erhalten wir
\[ab=a\cdot b + a\wedge b = |a||b|\cos\alpha + |a||b|\sin\alpha\*i =
|a||b|(\cos\alpha + \sin\alpha\*i)\]
Kaum haben wir $\*i$ entdeckt, schon winkt uns $e^{\alpha\*i}$ entgegen, wenn
wir $e^{\alpha\*i} := \cos\alpha + \*i\sin\alpha$ setzen. Ist diese Definition
sinnvoll? Waren wir nicht zu schnell? Betrachten wir den Anfang der
Taylorreihe von $e^{\alpha\*i}$
\[e^{\alpha\*i}=1+\frac{\alpha\*i}{1!}+\frac{(\alpha\*i)^2}{2!}
+\frac{(\alpha\*i)^3}{3!}+\frac{(\alpha\*i)^4}{4!}+\frac{(\alpha\*i)^5}{5!}+ ...\]
Oh Gott, jetzt müssen wir auch noch Potenzen von $\alpha\*i$ berechnen. Na
dann, ran an die Arbeit. Da $\alpha$ eine reelle Zahl ist, reicht nur die
Potenzen von $\*i$ zu betrachten.
\begin{enumerate}
    \item $(\*i)^1=\*i$ done.
    \item $(\*i)^2=(e_1\wedge e_2)^2=-1$ auch durch.
    \item $(\*i)^3 = (\*i)^2\*i = -\*i$. Mensch das geht ja schnell.
    \item $(\*i)^4=(\*i)^2(\*i)^2=1$. Da sage ich nix zu. :)
    \item $(\*i)^5=(\*i)^3(\*i)^2=\*i$. Ok, Muster erkannt.
\end{enumerate}
Setzen wir das ein, so kommt
\[e^{\alpha\*i}=1+\frac{\alpha\*i}{1!}-\frac{(\alpha)^2}{2!}
+\frac{(\alpha)^3\*i}{3!}+\frac{(\alpha)^4}{4!}+\frac{(\alpha)^5\*i}{5!}+ ...\]
raus. Sortieren wir die Summanden in der Reihe um und sagen das alle
notwendigen Voraussetzungen bzgl der absoluten Konvergenz erfüllt sind, so
erhalten wir
\[e^{\alpha\*i}=\left(1-\frac{(\alpha)^2}{2!}+\frac{(\alpha)^4}{4!}-...\right)
+\*i\left(\frac{\alpha}{1!}+\frac{(\alpha)^3}{3!}+\frac{(\alpha)^5}{5!}+
...\right)=\cos\alpha+\*i\sin\alpha\]
Damit ist die Definition von $e^{\alpha\*i}$ sinnvoll. Was können wir damit
anfangen? Um das zu verstehen gehen wir ein Schritt zurück und betrachten die
zwei Seiten einer Medaille bzw. dem Bivektor $\*i$. Die eine Seite ist $\*i$
als orientierte Fläche zu betrachten. Zeichnen wir ein Koordinatensystem der
Ebene auf mit dem Bivektor $\*i$.
\shorthandoff{"}
\begin{center}
\begin{minipage}{\linewidth}
\centering
\begin{tikzpicture}[>=Stealth]
    \coordinate (O)  at (0.0, 0.0);
    \coordinate (e1) at (1.0, 0.0);
    \coordinate (e2) at (0.0, 1.0);

    \draw[step=1.0, lightgray, very thin] (-0.8,-0.8) grid (4.8, 2.8);
    \draw ( 0.0, -1.0) -- (0.0, 3.0);
    \draw (-1.0,  0.0) -- (5.0, 0.0);
    \draw [draw=black, fill=green, opacity=0.2]
    (O) -- (e1) -- ($(e1) + (e2)$) -- (e2) -- cycle;

    \draw[->, thick] (O) -- (e1);
    \draw[->, thick] (e1) -- ($(e1) + (e2)$);
    \node[label=below:{$e_1$}] at (0.5, 0.0) {};
    \node[label=right:{$e_2$}] at (1.0, 0.5) {};
\end{tikzpicture}
\captionof{figure}{Der Bivektors $\*i$ in der Ebene}
\label{fig:bivector1}
\end{minipage}
\end{center}
\shorthandon{"}
Zeichnen wir noch zwei Vektoren $a=3e_1+0e_2$ und $b=1e_1+2e_2$ ein und
betrachten das äußere Produkt von $a\wedge b$ so sieht das dann so aus
\shorthandoff{"}
\begin{center}
\begin{minipage}{\linewidth}
\centering
\begin{tikzpicture}[>=Stealth]
    \coordinate (O)  at (0.0, 0.0);
    \coordinate (e1) at (1.0, 0.0);
    \coordinate (e2) at (0.0, 1.0);
    \coordinate (a)  at (3.0, 0.0);
    \coordinate (b)  at (1.0, 2.0);
    \draw [draw=black, fill=green, opacity=0.2]
    (O) -- (e1) -- ($(e1) + (e2)$) -- (e2) -- cycle;

    \draw[step=1.0, lightgray, very thin] (-0.8,-0.8) grid (4.8, 2.8);
    \draw ( 0.0, -1.0) -- (0.0, 3.0);
    \draw (-1.0,  0.0) -- (5.0, 0.0);
    \draw [draw=black, fill=green, opacity=0.2]
    (O) -- (a) -- ($(a) + (b)$) -- (b) -- cycle;

    \draw[->, thick] (O) -- (e1);
    \draw[->, thick] (e1) -- ($(e1) + (e2)$);
    \node[label=below:{$e_1$}] at (0.5, 0.0) {};
    \node[label=right:{$e_2$}] at (1.0, 0.5) {};

    \draw[->, thick] (O) -- (a);
    \draw[->, thick] (a) -- ($(a)+(b)$);
    \node[label=below:{$a$}] at (2.5, 0.0) {};
    \node[fill=white] at (3.9, 1.0) {$b$};
\end{tikzpicture}
\captionof{figure}{Der Bivektors $\*i$ in der Ebene}
\label{fig:bivector2}
\end{minipage}
\end{center}
\shorthandon{"}
Wenden wir den Gummi-Stab Trick so an, dass das neue Parallelogramm ein Rechteck wird.
\shorthandoff{"}
\begin{center}
\begin{minipage}{\linewidth}
\centering
\begin{tikzpicture}[>=Stealth]
    \coordinate (O)  at (0.0, 0.0);
    \coordinate (e1) at (1.0, 0.0);
    \coordinate (e2) at (0.0, 1.0);
    \coordinate (a)  at (3.0, 0.0);
    \coordinate (b)  at (0.0, 2.0);
    \draw [draw=black, fill=green, opacity=0.2]
    (O) -- (e1) -- ($(e1) + (e2)$) -- (e2) -- cycle;

    \draw[step=1.0, lightgray, very thin] (-0.8,-0.8) grid (4.8, 2.8);
    \draw ( 0.0, -1.0) -- (0.0, 3.0);
    \draw (-1.0,  0.0) -- (5.0, 0.0);
    \draw [draw=black, fill=green, opacity=0.2]
    (O) -- (a) -- ($(a) + (b)$) -- (b) -- cycle;

    \draw[->, thick] (O) -- (e1);
    \draw[->, thick] (e1) -- ($(e1) + (e2)$);
    \node[label=below:{$e_1$}] at (0.5, 0.0) {};
    \node[label=right:{$e_2$}] at (1.0, 0.5) {};
    \draw[->, thick] (O) -- (a);
    \draw[->, thick] (a) -- ($(a)+(b)$);
    \node[label=below:{$a$}] at (2.5, 0.0) {};
    \node[label=right:{$b'$}] at (3.0, 1.5) {};
\end{tikzpicture}
\captionof{figure}{Der Bivektors $\*i$ in der Ebene}
\label{fig:bivector3}
\end{minipage}
\end{center}
\shorthandon{"}
Jetzt können wir ganz geschmeidig die Größe der Fläche $a\wedge b$ abzählen
und erhalten mit der selben Orientierung wie $e_1\wedge e_2$
\[a\wedge b= 6e_1\wedge e_2\]

Ist es möglich, das Ergebnis auch zu berechnen? Ja!
\begin{align*}
    a\wedge b &= 3e_1\wedge(e_1+2e_2) \\
    &= 3e_1\wedge e_1 + 3e_1\wedge 2e_2 \\
    &=6 e_1\wedge e_2
\end{align*}
Wenn wir $b\wedge a$ genommen hätten, dann wäre $b\wedge a = -6e_1\wedge e_2$
rausgekommen. Somit ist $\*i=e_1e_2=e_1\wedge e_2$ \textbf{die} orientierte
Fläche mit der Größe eins. Anders gesagt $\*i$ ist \textbf{der} Bivetor der
Ebene. Es kann nur einen (in der Ebene) geben (Highlander).
\newpage
Kehren wir die Seite der Medaille um. Was stellt $\*i$ mit dem Vektor $e_1$
an, wenn wir $e_1$ mit $\*i$ von rechts multiplizieren?
\[e_1\*i\stackrel{1}=e_1(e_1e_2)\stackrel{2}=(e_1e_1)e_2\stackrel{3}=e_2\]
\begin{enumerate}
    \item Deifinition von $\*i$.
    \item Assoziativität des geometrischen Produktes.
    \item $e_1e_1=1$
\end{enumerate}
Was macht $\*i$ mit $e_1$, wenn von links multipliziert wird? Die Rechnung
mache wir etwas schneller, da wir schon geübter sind. :)
\[\*ie_1=(e_1e_2)e_1=(-e_2e_1)e_1=-e_2\]
Die Multiplikation mit $\*i$ von rechts führt $e_1$ in $e_2$ und von links in
$-e_2$. Das ganze noch für $e_2$.
\begin{align*}
    e_2\*i &= e_2(e_1e_2) = e_2(-e_2e_1) = -e_1 \\
    \*ie_2 &= (e_1e_2)e_2 = e_1(e_2e_2) = e_1
\end{align*}

Durch diese Rechnungen ist jetzt die Multiplikation von $\*i$ mit einem
beliebigen Vektor $a=xe_1+ye_2$ schnell gemacht.
\begin{align*}
    a\*i &= (xe_1+ye_2)\*i = xe_1\*i+ye_2\*i = -ye_1 + xe_2 \\
    \*ia &= \*i(xe_1+ye_2) = x\*ie_1+y\*ie_2 = +ye_1 - xe_2
\end{align*}
Zeichen wir das ganze zum Beispiel mit dem Vektor $a=3e_1+2e_2$ auf
\shorthandoff{"}
\begin{center}
\begin{minipage}{\linewidth}
\centering
\begin{tikzpicture}[>=Stealth]
    \coordinate (O)  at ( 0.0,  0.0);
    \coordinate (e1) at ( 1.0,  0.0);
    \coordinate (e2) at ( 0.0,  1.0);
    \coordinate (a)  at ( 3.0,  2.0);
    \coordinate (a1) at (-2.0,  3.0);
    \coordinate (a2) at ( 2.0, -3.0);

    \draw[step=1.0, lightgray, very thin] (-3.8,-3.8) grid (3.8, 3.8);
    \draw ( 0.0, -4.0) -- (0.0, 4.0);
    \draw (-4.0,  0.0) -- (4.0, 0.0);

    \draw[->, thick] (O) -- (e1);
    \draw[->, thick] (O) -- (e2);
    \draw[->, thick, blue] (O) -- (a);
    \draw[->, thick, red] (O) -- (a1);
    \draw[->, thick, green] (O) -- (a2);
    \node[label=below:{$e_1$}] at (1.0, 0.0) {};
    \node[label=right:{$e_2$}]  at (0.0, 1.0) {};
    \node[label=below:{$a$}] at (2.7, 1.8) {};
    \node[label=above:{$a\*i$}] at (-1.5, 2.5) {};
    \node[label=right:{$\*ia$}] at (1.6, -2.4) {};
    \pic["$\*i$ von rechts", ->, draw=red, angle radius=2.0cm, angle eccentricity=1.2]
    {angle=a--O--a1};
    \pic["$\*i$ von links", <-, draw=green, angle radius=2.0cm, angle eccentricity=1.6]
    {angle=a2--O--a};
\end{tikzpicture}
\captionof{figure}{Rotation um $90^\circ$ mit $\*i$}
\label{fig:rotate90degree}
\end{minipage}
\end{center}
\shorthandon{"}

Zusammenfassend sind die beiden Seiten der Medaille
\begin{itemize}
    \item $\*i$ ist der Einheitsbivektor.
    \item $\*i$ rotiert die Vektoren der Ebene um $90^{\circ}$.
\end{itemize}

Nach diesen Erkenntnissen wenden wir $e^{\alpha\*i}$ auf einen beliebigen
Vektor $a$ an.
\begin{align*}
    e^{\alpha\*i} a &= (\cos\alpha + \*i \sin\alpha) a = a \cos\alpha +
    \*ia \sin\alpha \\
    a e^{\alpha\*i} &= a (\cos\alpha + \*i \sin\alpha) = a \cos\alpha +
    a\*i \sin\alpha
\end{align*}

Weiter oben haben wir die Werte für $a\*i$ und $\*ia$ schon berechnet. Setzen
wir dies ein und erhalten
\begin{align*}
    e^{\alpha\*i} a &= (xe_1 +ye_2) \cos\alpha + (ye_1 -xe_2) \sin\alpha \\
    &= (x\cos\alpha+y\sin\alpha)e_1 + (y\cos\alpha - x\sin\alpha) e_2 \\
    a e^{\alpha\*i} &= (xe_1 -ye_2)\cos\alpha + (ye_1 + xe_2) \sin\alpha \\
    &= (x\cos\alpha-y\sin\alpha)e_1 + (y\cos\alpha +x\sin\alpha) e_2
\end{align*}

Das alles sieht nache einer Drehung des Vektors $a$ um den Winkel $\alpha$
aus. Im Falle von $ae^{\alpha\*i}$ gegen dem Uhrzeigersinn und im Falle von
$e^{\alpha\*i}a$ mit dem Uhrzeigersinn. Wie sind wir auf die Formel mit
$e^{\alpha\*i}$ gekommen? Wir hatten das geometrische Produkt von zwei
Vektoren betrachtet. Die hießen dort $a$ und $b$. Da aber $a$ ein
vielbeschfäftigter Vektor  ist nenne ich sie mal in $u$ und $v$ um. Damit ist
dann die Formel
\[
    uv=|u||v|e^{\alpha\*i}
\]
Wobei $\alpha$ der Winkel zwischen den Vektoren $u$ und $v$ ist. Haben $u$ und
$v$ die Länge $1$ so ist
\[
    uv=e^{\alpha\*i}
\]
Dieses geometrische Produkt von Einheitsbivektoren wird als Rotor bezeichnet.
Setzen wir $R_\alpha := e^{\alpha\*i}$ und spielen mit der Gleichung
$uv=R_\alpha$.  Multiplizieren sie einmal links mit $u$ und einmal rechts mit
$v$ so erhalten wir die Gleichungen
\[
    v = uR_\alpha\hspace{3ex}u = R_\alpha v
\]
Die erste Gleichung führt den Vektor $u$ in $v$ und die zweite Gleichung $v$ in
$u$ über. Der Rotor $R_\alpha$ führt also tatsache eine Rotation der
Einheitsvektoren durch. Damit haben wir die koordinatenfreie Interpretation des
Rotors $R_\alpha$ erhalten.  Schauen wir paar Bilder dazu
an.
\shorthandoff{"}
\begin{center}
\begin{minipage}{\linewidth}
\centering
\begin{tikzpicture}[>=Stealth]
    \coordinate (O) at (0.0, 0.0);
    \coordinate (u) at (2.0, 0.0);
    \coordinate (v) at ($(O)!1!40:(u)$);

    \draw[->, draw=red, thick] (O) -- (u);
    \draw[->, draw=red, thick] (O) -- (v);
    \pic["$\alpha$", -, draw=black, angle radius=2.0cm, angle eccentricity=1.2]
    {angle=u--O--v};
    \node at (1.0, -0.3) {$u$};
    \node at (0.7,  0.9) {$v$};

    \begin{scope}[shift={(3.0,0.0)}]
        \coordinate (O) at (0.0, 0.0);
        \coordinate (u) at (2.0, 0.0);
        \coordinate (v) at ($(O)!1!40:(u)$);

        \draw[-, thick] (O) -- (u);
        \draw[-, thick] (O) -- (v);
        \pic["$R_\alpha$", ->, thick, draw=red, angle radius=2.0cm, angle eccentricity=1.2]
    {angle=u--O--v};
    \end{scope}
\end{tikzpicture}
\captionof{figure}{$R_\alpha$ als gerichtetes Kreissegment}
\label{fig:rotatealpha}
\end{minipage}
\end{center}
\shorthandon{"}
Genauso wie wir einen Vektor $a$ als den gleichen Vektor betrachten, wenn wir
ihn in der Ebene verschieben, wird der Rotor $R_\alpha$ als der gleiche Rotor
der Ebene betrachtet, wenn er gedreht wird.
\shorthandoff{"}
\begin{center}
\begin{minipage}{\linewidth}
\centering
\begin{tikzpicture}[>=Stealth]
    \coordinate (O) at (0.0, 0.0);
    \coordinate (u) at (2.0, 0.0);
    \coordinate (v) at ($(O)!1!40:(u)$);

    \draw[-, thick] (O) -- (u);
    \draw[-, thick] (O) -- (v);
    \pic["$R_\alpha$", ->, thick, draw=red, angle radius=2.0cm, angle eccentricity=1.2]
    {angle=u--O--v};
    \begin{scope}[rotate=60]
        \coordinate (O) at (0.0, 0.0);
        \coordinate (u) at (2.0, 0.0);
        \coordinate (v) at ($(O)!1!40:(u)$);

        \draw[-, thick] (O) -- (u);
        \draw[-, thick] (O) -- (v);
        \pic["$R_\alpha$", ->, thick, draw=red, angle radius=2.0cm, angle eccentricity=1.2]
    {angle=u--O--v};
    \end{scope}
    \begin{scope}[rotate=150]
        \coordinate (O) at (0.0, 0.0);
        \coordinate (u) at (2.0, 0.0);
        \coordinate (v) at ($(O)!1!40:(u)$);

        \draw[-, thick] (O) -- (u);
        \draw[-, thick] (O) -- (v);
        \pic["$R_\alpha$", ->, thick, draw=red, angle radius=2.0cm, angle eccentricity=1.2]
    {angle=u--O--v};
    \end{scope}
    \begin{scope}[rotate=280]
        \coordinate (O) at (0.0, 0.0);
        \coordinate (u) at (2.0, 0.0);
        \coordinate (v) at ($(O)!1!40:(u)$);

        \draw[-, thick] (O) -- (u);
        \draw[-, thick] (O) -- (v);
        \pic["$R_\alpha$", ->, thick, draw=red, angle radius=2.0cm, angle eccentricity=1.2]
    {angle=u--O--v};
    \end{scope}
\end{tikzpicture}
\captionof{figure}{$R_\alpha$ als gerichtetes Kreissegment}
\label{fig:rotatealpha}
\end{minipage}
\end{center}
\shorthandon{"}
\newpage
Welche Objekte haben wir in der Ebene kennen gelernt? Wir wissen was passiert,
wenn wir eine reelle Zahl $\lambda$ mit einer anderen reellen Zahl $\mu$
multiplizieren. Es kommt eine neue reellen Zahl raus. Wir wissen auch was
$\lambda$ mit einem Vektor anstellt. Sie kürzt, verlängert oder dreht die
Richtung des Vektors um. Anders gesagt, sie veränder die Größe oder die
Orientierung des Vektors. Weiterhin stellt $\lambda$ mit $\*i$ das gleiche an,
wie mit einem Vektor. Sie verändert die Größe oder die Orientierung des
Bivektors $\*i$.

Ein Vektor mit einem anderen Vektor multipliziert, ergibt eine Zahl und ein
vielfaches des Bivektors $\*i$. Ein Vektor mit dem Bivektor $\*i$
multipliziert, ergibt ein (gedrehtes) Vektor. Schließlich ergibt der Bivektor
$\*i$ mit sich selbst multipliziert die Zahl $-1$.

Damit haben wir die Objekte $1, e_1, e_2$ und $e_1e_2$ der Ebene
Identifiziert. Werden diese Objekte in Diamantform strukturiert, so können
wir der einzelnen Ebenen, die bei der Strukturierung entsteht, Namen geben.
\[
\begin{matrix}
    1 \\ 
    e_1\hspace{3ex} e_2 \\ 
    e_1e_2
\end{matrix}
\hspace{3ex}
\begin{matrix*}[l]
    \textrm{basis für }0\textrm{-Vektoren} \\ 
    \textrm{basis für }1\textrm{-Vektoren} \\ 
    \textrm{basis für }2\textrm{-Vektoren}
\end{matrix*}
\]
Durch die Feststellung das es in der Ebene nur ein Bivektor mit der Größe eins
gibt, haben die Ebenen der Diamantform auch andere Namen.
\[
\begin{matrix}
    1 \\ 
    e_1\hspace{3ex} e_2 \\ 
    e_1e_2
\end{matrix}
\hspace{3ex}
\begin{matrix*}[l]
    \textrm{Skalar} \\ 
    \textrm{Vektoren} \\ 
    \textrm{Pseudoskalar}
\end{matrix*}
\]
Mit diesen Objekten können wir, durch Linearkombination dieser, ein sogenanntes
Multivektor erstellen.
\[
    m = \lambda_0 + \lambda_1e_1 + \lambda_2e_2 +
    \lambda_3e_1e_2\hspace{3ex}\textrm{für alle
    }\lambda_i\in\mathbb{R}\textrm{ mit } i=0,1,2,3
\]
\end{document}
