\documentclass[fleqn]{scrartcl}
\usepackage{amsmath}
\usepackage{tikz}
\usepackage[ngerman]{babel}
\usepackage[utf8]{inputenc}
\usetikzlibrary {arrows.meta, angles, quotes}

% Numerierung der Formel innerhalb einer Sektion
\numberwithin{equation}{section}

% Abkürzung für mathbf
\def\*#1{\mathbf{#1}}

\title{Geometrische Algebra \\
für Oliver Hut}
\author{Janos Sarközi}

\begin{document}
\maketitle
\newpage
\section{Historisches}
Die Idee der geometrischen Algebra war schon 1844 Hermann Grassmann in seinem
Buch über die Ausdehnungslehre entwickelt worden. Parallel dazu hatte sich
Willima Rowan Hamilton mit der Drehung von Objekten im Raum beschäftigt.
Initial erkannte Hamilton, dass die Drehung von Objekten in der Ebene mit den
komplexen Zahlen erreicht werden kann. Aus seiner Arbeit von Drehungen von
Objekten im Raum entstanden die Quaternionen. Seine berühmte Formel
\[i^2=j^2=k^2=ijk=-1\]
ist in Dublin auf der Boomer Bridge auf einer Tafel verewigt worden. Dort
hatte er den Geistesblitz während eines Spazierganges mit seiner Frau.
Grassmann und Hamilton wussten möglicher Weise von der Arbeit des anderen,
aber erkannten nicht die Ähnlichkeiten. Nur ein junger Mathematiker namens
William Kingdon Clifford erkannte dies. Da er schon mit 33 gestorben war,
er konnte sich nicht in der Welt der Mathematiker etablieren, wurde seine
Arbeit keine große Aufmerksamkeit gewidmet.

Über hundert Jahre vergingen, bis sich jemand wieder mit diesen Themen
befasste. Er nahm die Arbeiten von Grassman, Hamilton und Clifford auf und
führte die geometrische Algebra auf neue gefilde. Diese Person ist David
Hestenes.

\section{Motivation und Ideen}
Die Seele der ganzen geometrische Algebra lässt sich in der
Gleichung
\[ab = a\cdot b + a\wedge b\]
erkennen. Heutige Bezeichnung von $a$ und $b$ sind Vektoren. Dabei bedeutet
$a\cdot b$ das Skalarprodukt oder das innere Produkt und $a\wedge b$ das
äußere Produkt. Die Formel besagt also, dass das geometrische Produkt zweier
Vektoren $a$ und $b$ sich aus dem inneren und äußeren Produkt berechnet
werden kann. Es wird kein neues Zeichen für das geometrische Produkt
eingeführt.

Verlangt wurde von diesem geometrischen Produkt, dass sie assoziativ und 
distributiv ist
\begin{align*}
    a(bc) &= (ab)c \\
    a(b+c) &= ab + ac
\end{align*}
Wenden wir uns nun den einzelnen Komponenten des geometrischen Produktes zu.
\newpage
\subsection{Das innere Produkt}
Das innere Produkt ist uns allen durch die Formel bekannt
\[a\cdot b = |a||b|\cos\alpha \]
Wobei $\alpha$ der Winkel zwischen den Vektoren $a$ und $b$ ist. Visuell sieht
das so aus.
\shorthandoff{"}
\begin{center}
\begin{minipage}{\linewidth}
\centering
\begin{tikzpicture}[>=Stealth]
    \coordinate (O) at (0,0);
    \coordinate (A) at (3,0);
    \coordinate (B) at (1,1.5);

    \draw[->, thick] (O) -- (A);
    \draw[->, thick] (O) -- (B);
    \node[shift={(-0.5,-0.3)}] (v) at (B) {$b$};
    \node[shift={(-0.5,-0.3)}] (u) at (A) {$a$};
    \pic["$\alpha$", -, draw=black, angle radius=0.8cm, angle eccentricity=0.6] {angle=A--O--B};
\end{tikzpicture}
\captionof{figure}{Das Skalarprodukt zwischen den Vektoren $a$ und $b$}
\label{fig:innerProduct}
\end{minipage}
\end{center}
\shorthandon{"}

\subsection{Das äußere Produkt}
Das äußere Produkt der Vektoren $a$ und $b$ soll die orientierte Fläche des
Parallelogramms werden, das durch diese Vektoren aufgespannt wird. Hier ein
Bild
\shorthandoff{"}
\begin{center}
\begin{minipage}{\linewidth}
\centering
\begin{tikzpicture}[>=Stealth]
    \coordinate (O) at (0,0);
    \coordinate (A) at (3,0);
    \coordinate (C) at (4,1.5);
    \coordinate (B) at (1,1.5);

    \filldraw[fill=green!20] (O) -- (A) -- (C) -- (B);
    \draw[->, thick] (O) -- (A);
    \draw[->, thick] (O) -- (B);
    \draw (B) -- (C);
    \draw (A) -- (C);
    \node[shift={(-0.5,-0.3)}] (v) at (B) {$b$};
    \node[shift={(-0.5,-0.3)}] (u) at (A) {$a$};
    \pic ["$\alpha$", -, draw=black, angle radius=0.8cm, angle eccentricity=0.6] {angle=A--O--B};
    \draw[->, semithick] ([shift=(0:0.3)]1.8,0.8) arc (0:270:0.3);
\end{tikzpicture}
\captionof{figure}{Das äußere Produkt zwischen den Vektoren $a$ und $b$}
\label{fig:outherProduct}
\end{minipage}
\end{center}
\shorthandon{"}

Die Orientierung der Fläche wird durch die Reihenfolge der Vektoren $a$ und
$b$ in dem äußeren Produkt $a\wedge b$ bestimmt und wird mit einem kleinen
Kreissegment mit Pfeil gekennzeichnet. Wird das äußere Produkt von $b$ mit $a$
gebildet, so soll eine andere Orientierung enstehen. Da es nur zwei
Orientierungen gibt wird die Beziehung
\[a\wedge b = - b\wedge a\]
zwischen den äußeren Produkten gelten. Die Fläche des Parallelogramms lässt
sich mit der Gleichung
\[|a\wedge b| = |a||b|\sin\alpha\]
beschreiben.

Hier enstehen die Ähnlichkeiten zwischen einem Vektor und einer orientierten
Fläche. Beide haben eine Größe und eine Orientierung. Die Größe nennt sich bei
dem Vektor die Länge und bei dem äußeren Produkt die Fläche. Die Orientierung
bei der Fläche hat auch den Namen links bzw. rechts orientiert oder mit dem
Uhrzeigersinn bzw. gegen dem Uhrzeigersinn.

Paar interessante Eigenschaften des äußeren Produktes. Zuerst eine einfache.
Liegt der Vektor $b$ in der Richtung von $a$ so ist das äußere Produkt Null.
Wenn also $b=\lambda a$ ist, so soll
\[b\wedge a = (\lambda a)\wedge a= 0\]
sein. Das leuchtet ein, da dann die Fläche des Parallelogramms Null ist. Das
folgt aus der genannten Gleichung über die Orientierung der Fläche
\[a\wedge b = - b\wedge a \textrm{ ist äquivalent zu } a\wedge b + b\wedge a = 0\]
Setzen wir $b = a$ ein, so kommt
\[2(a\wedge a)=0\]
raus und damit
\[a\wedge a = 0\]

Da das äußere Produkt als ein Produkt bezeichnet wird, soll auch die
Distributivität gelten.
\[a\wedge (b + c) = a\wedge b + a\wedge c\]

Nun können wir uns auf die Gleichheit von orientierten Fläche stürzen. Wird
die Spitze des Vektor $b$ parallel zu dem Vektor $a$ verschoben, so ändert
sich die Fläche des Parallelogramms nicht. Dies folgt aus den vorherigen
Bemerkungen. Sei dafür $b'= b + \lambda a$. Dann gilt
\[a\wedge b' = a\wedge (b + \lambda a) = a \wedge b + a\wedge (\lambda a)=
a\wedge b\]
\shorthandoff{"}
\begin{center}
\begin{minipage}{\linewidth}
\centering
\begin{tikzpicture}[>=Stealth]
    \coordinate (O) at (0,0);
    \coordinate (A) at (3,0);
    \coordinate (C) at (4,1.5);
    \coordinate (B) at (1,1.5);

    \filldraw[fill=green!20] (O) -- (A) -- (C) -- (B);
    \draw[->, thick] (O) -- (A);
    \draw[->, thick] (O) -- (B);
    \draw (B) -- (C);
    \draw (A) -- (C);
    \node[shift={(-0.5,-0.3)}] (v) at (B) {$b$};
    \node[shift={(-0.5,-0.3)}] (u) at (A) {$a$};
    \draw[->, semithick] ([shift=(0:0.3)]1.8,0.8) arc (0:270:0.3);

    \begin{scope}[shift={(5.5,0)}]
        \coordinate (O') at (0,0);
        \coordinate (A') at (3,0);
        \coordinate (C') at (2.7,1.5);
        \coordinate (B') at (-0.3,1.5);

        \filldraw[fill=green!20] (O') -- (A') -- (C') -- (B');
        \draw[->, thick] (O') -- (A');
        \draw[->, thick] (O') -- (B');
        \draw (B') -- (C');
        \draw (A') -- (C');
        \node[shift={(-0.2,-0.3)}] (v) at (B') {$b'$};
        \node[shift={(-0.5,-0.3)}] (u) at (A') {$a$};
        \draw[->, semithick] ([shift=(0:0.3)]1.3,0.8) arc (0:270:0.3);
    \end{scope}
\end{tikzpicture}
\captionof{figure}{Verschiebung der Spitze des Vektors $b$ parallel zu $a$}
\label{fig:shiftb}
\end{minipage}
\end{center}
\shorthandon{"}

Das gleiche gilt auch für den Vektor $a$, wenn die Spitze des Vektors $a$
parallel zu dem Vektor $b$ verschoben wird. Eins weiter gedacht, gilt dies auch
für den Vektor $b'$ und $a$. Wenn also die Spitze des Vektors $a$ parallel zu
dem Vektor $b'$ verschoben wird. 
%\newpage
Hier das Endergebnis unsrer Überlegungen.
\shorthandoff{"}
\begin{center}
\begin{minipage}{\linewidth}
\centering
\begin{tikzpicture}[>=Stealth]
    \coordinate (O) at (0,0);
    \coordinate (A) at (3,0);
    \coordinate (C) at (4,1.5);
    \coordinate (B) at (1,1.5);

    \filldraw[fill=green!20] (O) -- (A) -- (C) -- (B);
    \draw[->, thick] (O) -- (A);
    \draw[->, thick] (O) -- (B);
    \draw (B) -- (C);
    \draw (A) -- (C);
    \node[shift={(-0.5,-0.3)}] (v) at (B) {$b$};
    \node[shift={(-0.5,-0.3)}] (u) at (A) {$a$};
    \draw[->, semithick] ([shift=(0:0.3)]1.8,0.8) arc (0:270:0.3);

    \begin{scope}[shift={(5.5,0)}]
        \coordinate (O') at (0,0);
        \coordinate (A') at (2.9,0.5);
        \coordinate (C') at (2.6,2.0);
        \coordinate (B') at (-0.3,1.5);

        \filldraw[fill=green!20] (O') -- (A') -- (C') -- (B');
        \draw[->, thick] (O') -- (A');
        \draw[->, thick] (O') -- (B');
        \draw (B') -- (C');
        \draw (A') -- (C');
        \node[shift={(-0.2,-0.3)}] (v) at (B') {$b'$};
        \node[shift={(-0.5,-0.3)}] (u) at (A') {$a'$};
        \draw[->, semithick] ([shift=(0:0.3)]1.2,1.0) arc (0:270:0.3);
    \end{scope}
\end{tikzpicture}
\captionof{figure}{Verschiebungen der Vektoren $a$ und $b$}
\label{fig:shiftshift}
\end{minipage}
\end{center}
\shorthandon{"}

Die Flächen sind gleich, die Orientierung ist gleich ... Ahaaa ... Ich kann
also beliebige Flächen betrachten, solange nur die Orientierung und Größe der
Flächen gleich ist ändert sich das äußere Produkt nicht. Somit kann das äußere
Produkt als ein Kreis, eine Ellipse oder ein wabbeliges Ding dargestellt
werden.

\subsection{Eingenschaften des geometrischen Produktes}
Durch die eben beschriebenen Eingenschaften des inneren und äußeren Produktes,
können wir paar Eingenschaften des geometrischen Produktes herleiten. Aus der
Gleichung
\[ab = a\cdot b + a\wedge b\]
können wir folgern
\begin{itemize}
\item Wenn $a \cdot b = 0$, dann $ab = a \wedge b$.
  \begin{itemize}
  \item Das heißt, wenn $a \perp b$, dann $ab = a\wedge b$.
  \item Weiterhin gilt auch wenn $a \perp b$, dann $ab = a \wedge b = - b
      \wedge a = -ba$
  \end{itemize}
\item Wenn $a\wedge b = 0$, dann $ab = a \cdot b$.
  \begin{itemize}
  \item Das heißt, wenn $a \parallel b$, dann $ab = a \cdot b$.
  \end{itemize}
\end{itemize}

Aber halt ... was haben wir gemacht? Eine Zahl $a\cdot b$ mit einer orientierten
Fläche $a\wedge b$ addiert? Haben wir nicht Äpfel mit Birnen verglichen bzw.
addiert? Wenn wir ehrlich sind, dann machen wir das gleiche bei den komplexen
Zahlen. Denn ist $z$ eine komplexe Zahl, so schreiben wir ganz mutig
\[z=x+yi\]
wobei $x$ und $y$ reelle Zahlen sind und $i$ die berühmte imaginäre Zahl ist.
Diese imaginäre Zahl multiplizieren wir sogar mit einer reellen Zahl und
addieren das ganze zu einer reellen Zahl?! Da mischen wir auch Äpfel mit
Birnen und das geht sogar recht gut! Also lass uns eine reelle Zahl $a\cdot b$
mit einer orientierten Fläche $a\wedge b$ addieren und schauen was rauskommt.

Ist also $a \parallel b$, dann ist, wie wir oben fest gestellt haben, $ab = a
\cdot b$. Setzen wir $b=a$ ein, dann ist
\[aa=a\cdot a = |a|^2\]
Teilen wir beide Seiten mit $|a|^2$, so erhalten wir
\[\frac{aa}{|a|^2}=1.\]
Wenn wir jetzt noch zulassen, das wir die positive reelle Zahl $|a|^2$ auf
eines der Vektoren verteilen dürfen, dann erhalten wir
\[a\frac{a}{|a|^2}=1\]
Durch ein scharfes hinsehen erkennen wir das $\frac{a}{|a|^2}$ zu $a$ invers
ist. Denn das geometrische Produkt von $a$ mit $\frac{a}{|a|^2}$ ist eins. Wow
... wir dürfen mit einem Vektor teilen?! Das ist neu!
\newpage
Wie meine ist das mit einem Vektor teilen. Jeder Vektor (ungleich Null) hat
eine Inverse bzgl. des geometrischen Produktes. Damit können wir zum Beispiel
die Gleichung
\[ab=c\]
umschreiben in
\[a=cb^{-1}=c\frac{b}{|b|^2}=c\frac{b}{b\cdot b}\]
und wenn wir cheaten können wir mit $b$ kürzen und erhalten
\[a=\frac{c}{b}\]
Diese Gleichung ist aber mit vorsicht zu genießen, da wir aus ihr nicht sehen,
ob wir die links oder die rechts Inverse von $b$ genommen haben!
\end{document}
