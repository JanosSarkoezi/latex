\documentclass[11pt, a4paper, fleqn]{report}
\usepackage{amsmath}
\usepackage[utf8]{inputenc}

% Numerierung der Fromel innerhalb einer Sektion
\numberwithin{equation}{section}

% Abkürzung für mathbf
\def\*#1{\mathbf{#1}}

\title{New Foundations for Classical Mechanics}

\begin{document}
\maketitle
\chapter{Origins of Geometric Algebra}
\section{Geometry and Physics}
\section{Number and Magnitude}
\section{Directed Numbers}
\begin{equation}
    \*a = \*b
\end{equation}
\begin{equation}
    \*b = \lambda\*a
\end{equation}
\begin{equation}
    \*a = |\*a|\hat{\*a}, \textrm{ where } |\hat{\*a}|=1
\end{equation}
\begin{equation}
    (0)\*a=0
\end{equation}
\begin{equation}
    \*a+\*b=\*s
\end{equation}
\begin{equation}
    \*a+\*b=\*b+\*a
\end{equation}
\begin{equation}
    (\*a+\*b)+\*c=\*a+(\*b+\*c)
\end{equation}
\begin{equation}
    \*a+0=\*a
\end{equation}
\begin{equation}
    \*a+\*b=0
\end{equation}
\begin{equation}
    \*c-\*a=\*c+(-\*a)
\end{equation}
\begin{equation}
    (-1)\*a=-\*a
\end{equation}



\section{The Inner Product}
\begin{equation}
    \*a\cdot\*b=|\*a||\*b|\cos\gamma
\end{equation}
\begin{equation}
    \*a \cdot \*b = \*b \cdot \*a
\end{equation}
\begin{equation}
    (\lambda\*a)\cdot\*b = \lambda(\*b \cdot\*a)=\*a \cdot(\lambda\*b)
\end{equation}
\begin{equation}
    \*a \cdot(\*b +\*c)=\*a \cdot\*b +\*a \cdot \*c
\end{equation}
\begin{equation}
    \*a \cdot \*a = |\*a|^2\geq 0
\end{equation}
\begin{equation}
    \*a + \*b=\*c
\end{equation}
\begin{equation}
    |\*c|^2=|\*a|^2+|\*b|^2+2\*a \cdot \*b
\end{equation}
\begin{equation}
    c^2=a^2+b^2+2ab\cos\gamma
\end{equation}
\begin{equation}
    \*a \cdot \*b=0
\end{equation}



\section{The Outer Product}
\begin{equation}
    \*a \wedge \*b=\*B
\end{equation}
\begin{equation}
    \*b\wedge\*a=-\*a \wedge \*b=-\*B
\end{equation}
\begin{equation}
    \*b \wedge \*a=\*a \wedge(-\*b)=(-\*b)\wedge(-\*a)=(-\*a)\wedge\*b
\end{equation}
\begin{equation}
    |\*B|=|\*a\wedge\*b|=|\*a||\*b|\sin\gamma
\end{equation}
\begin{equation}
    \*C=\lambda\*B
\end{equation}
\begin{equation}
    |\*C|=|\lambda||\*B|
\end{equation}
\begin{equation}
    (1)\*B=\*B\textrm{ and }(-1)\*B=-\*B
\end{equation}
\begin{equation}
    (\lambda\*a)\wedge\*b=\lambda(\*b\wedge\*a)=\*a\wedge(\lambda\*b)
\end{equation}
\begin{equation}
    \*a\wedge\*b=0
\end{equation}
\begin{equation}
    \*a\wedge\*a=0
\end{equation}
\begin{equation}
    \*a\wedge(\*b+\*c)=\*a\wedge\*b+\*a\wedge\*c
\end{equation}
\begin{equation}
    \*a\wedge\*c=\*a\wedge\*b=\*c\wedge\*b
\end{equation}
\begin{equation}
    |\*a\wedge\*c|=|\*a\wedge\*b|=|\*c\wedge\*b|
\end{equation}
\begin{equation}
    \frac{\sin\alpha}{a} =\frac{\sin\beta}{b}=\frac{\sin\gamma}{c}
\end{equation}
\begin{equation}
    (\*a\wedge\*b)\wedge\*c=\*T
\end{equation}
\begin{equation}
    (\*a\wedge\*b)\wedge\*c=\*a\wedge(\*b\wedge\*c)
\end{equation}
\begin{equation}
    (\*b\wedge\*a)\wedge\*c=(-\*a\wedge\*b)\wedge\*c=-\*T
\end{equation}
\begin{equation}
    (\*a\wedge\*b)\wedge\*c=\*a\wedge\*b\wedge\*c=0
\end{equation}
\begin{equation}
    (\*a\wedge\*b\wedge\*c)\wedge\*d=\*a\wedge\*b\wedge\*c\wedge\*d=0
\end{equation}



\section{Synthesis and Simplification}
\begin{equation}
    \*a\*b=\*a\cdot\*b+\*a\wedge\*b
\end{equation}
\begin{equation}
    \*b\*a=\*b\cdot\*a+\*b\wedge\*a=\*a\cdot\*b-\*a\wedge\*b
\end{equation}
\begin{equation}
    \*a\wedge\*b=0\rightarrow \*a\*b=\*a\cdot\*b=\*b\*a
\end{equation}
\begin{equation}
    \*a\cdot\*b=0\rightarrow \*a\*b=\*a\wedge\*b=-\*b\wedge\*a=-\*b\*a
\end{equation}
\begin{equation}
    \*a(\*b + \*c)=\*a\*b + \*a\*c
\end{equation}
\begin{equation}
    (\*b + \+c)\*a=\*b\*a + \*c\*a
\end{equation}
\begin{equation}
    (\lambda\*a)\*b=\lambda(\mathbf{b}\mathbf{a})=\mathbf{a}(\lambda\mathbf{b})
\end{equation}
\begin{equation}
    \lambda\mathbf{a}=\mathbf{a}\lambda
\end{equation}
\begin{equation}
    \mathbf{a}\cdot\mathbf{b}=\frac{1}{2}(\mathbf{a}\mathbf{b}+\mathbf{b}\mathbf{a})
\end{equation}
\begin{equation}
    \mathbf{a}\wedge\mathbf{b}=\frac{1}{2}(\mathbf{a}\mathbf{b}-\mathbf{b}\mathbf{a})
\end{equation}
\begin{equation}
    \mathbf{a}(\mathbf{b}\mathbf{c})=(\mathbf{a}\mathbf{b})\mathbf{c}=\mathbf{a}\mathbf{b}\mathbf{c}
\end{equation}
\begin{equation}
    \mathbf{a}\cdot\mathbf{B}=\frac{1}{2}(\mathbf{a}\mathbf{B}-\mathbf{B}\mathbf{a})=-\mathbf{B}\cdot\mathbf{a}
\end{equation}
\begin{equation}
    \mathbf{a}\wedge\mathbf{B}=\frac{1}{2}(\mathbf{a}\mathbf{B}+\mathbf{B}\mathbf{a})=\mathbf{B}\wedge\mathbf{a}
\end{equation}
\begin{equation}
    \mathbf{a}\mathbf{B}=\mathbf{a}\cdot\mathbf{B}+\mathbf{a}\wedge\mathbf{B}
\end{equation}
\begin{equation}
    \mathbf{a}\cdot(\mathbf{b}\wedge\mathbf{c})=(\mathbf{a}\cdot\mathbf{b})\mathbf{c}-(\mathbf{a}\cdot\mathbf{c})\mathbf{b}
\end{equation}



\section{Axioms for Geometric Algebra}
\begin{equation}
    A = A_0+\mathbf{A}_1+\mathbf{A}_2+\mathbf{A}_3
\end{equation}
\begin{equation}
    A+B=B+A
\end{equation}
\begin{equation}
    (A+B)+C=A+(B+C)
\end{equation}
\begin{equation}
    (AB)C=A(BC)
\end{equation}
\begin{equation}
    A(B+C)=AB+AC
\end{equation}
\begin{equation}
    (B+C)A=BA+CA
\end{equation}
\begin{equation}
    A+0=A
\end{equation}
\begin{equation}
    1A=A
\end{equation}
\begin{equation}
    A+(-A)=0
\end{equation}
\begin{equation}
    \lambda A=A\lambda
\end{equation}
\begin{equation}
    \mathbf{a}^2=|\mathbf{a}|^2>0
\end{equation}
\begin{equation}\tag{1.7.12a}
    \mathbf{a}\cdot\mathbf{A}_k=\frac{1}{2}(\mathbf{a}\mathbf{A}_k-(-1)^k\mathbf{A}_k\mathbf{a})
\end{equation}
\begin{equation}\tag{1.7.12b}
    \mathbf{a}\wedge\mathbf{A}_k=\frac{1}{2}(\mathbf{a}\mathbf{A}_k+(-1)^k\mathbf{A}_k\mathbf{a})
\end{equation}
\begin{equation}\tag{1.7.12c}
    \mathbf{a}\mathbf{A}_k=\mathbf{a}\cdot\mathbf{A}_k+\mathbf{a}\wedge\mathbf{A}_k
\end{equation}
\begin{equation}\tag{1.7.13a}
    \mathbf{a}\cdot\mathbf{A}_k\textrm{ is a (k-1) vector}
\end{equation}
\begin{equation}\tag{1.7.13b}
    \mathbf{a}\wedge\mathbf{A}_k\textrm{ is a (k+1) vector}
\end{equation}
\begin{equation}\tag{1.7.14a}
    \mathbf{a}\wedge\mathbf{A}_3=0
\end{equation}
\begin{equation}\tag{1.7.14b}
    \mathbf{a}\wedge\mathbf{A}_3=0\rightarrow\mathbf{a}\mathbf{A}_3=\mathbf{A}_3\mathbf{a}
\end{equation}
\setcounter{equation}{14}%
\begin{equation}
    \mathbf{A}^{-1}\mathbf{A}=1
\end{equation}
\begin{equation}
    \mathbf{a}^{-1}=\frac{\mathbf{a}}{|\mathbf{a}|^2}
\end{equation}
\begin{equation}
    \mathbf{A}^{-1}\mathbf{B}=:\frac{1}{\mathbf{A}}\mathbf{B}
\end{equation}
\begin{equation}
    \mathbf{B}\mathbf{A}^{-1}=:\mathbf{B}\frac{1}{\mathbf{A}}=:\mathbf{B}/\mathbf{A}
\end{equation}


\chapter{Developments in Geometric Algebra}
\section{Basic Identities and Definitions}
\begin{equation}
    \mathbf{a}\mathbf{b}=\mathbf{a}\cdot\mathbf{b}+\mathbf{a}\wedge\mathbf{b}
\end{equation}
\begin{equation}
    \mathbf{a}\cdot\mathbf{b}=\frac{1}{2}(\mathbf{a}\mathbf{b}+\mathbf{b}\mathbf{a})
\end{equation}
\begin{equation}
    \mathbf{a}\wedge\mathbf{b}=\frac{1}{2}(\mathbf{a}\mathbf{b}-\mathbf{b}\mathbf{a})
\end{equation}
\begin{equation}
    \mathbf{a}\mathbf{A}_r=\mathbf{a}\cdot\mathbf{A}_r+\mathbf{a}\wedge\mathbf{A}_r
\end{equation}
\begin{equation}
    \mathbf{a}\cdot\mathbf{A}_r=\frac{1}{2}(\mathbf{a}\mathbf{A}_r-(-1)^r\mathbf{A}_r\mathbf{a})=(-1)^{r+1}\mathbf{A}_r\cdot\mathbf{a}
\end{equation}
\begin{equation}
    \mathbf{a}\wedge\mathbf{A}_r=\frac{1}{2}(\mathbf{a}\mathbf{A}_r+(-1)^r\mathbf{A}_r\mathbf{a})=(-1)^{r}\mathbf{A}_r\wedge\mathbf{a}
\end{equation}
\[
    \mathbf{a}(\mathbf{b}\mathbf{c})=(\mathbf{a}\mathbf{b})\mathbf{c}
\]
\[
    \mathbf{a}(\mathbf{b}\cdot\mathbf{c}+\mathbf{b}\wedge\mathbf{c})=(\mathbf{a}\cdot\mathbf{b}+\mathbf{a}\wedge\mathbf{b})\mathbf{c}
\]
\[
    \mathbf{a}(\mathbf{b}\cdot\mathbf{c})+\mathbf{a}\cdot(\mathbf{b}\wedge\mathbf{c})+\mathbf{a}\wedge(\mathbf{b}\wedge\mathbf{c})=(\mathbf{a}\cdot\mathbf{b})\mathbf{c}+(\mathbf{a}\wedge\mathbf{b})\cdot\mathbf{c}+(\mathbf{a}\wedge\mathbf{b})\wedge\mathbf{c}
\]
\begin{equation}
    \mathbf{a}\wedge(\mathbf{b}\wedge\mathbf{c})=(\mathbf{a}\wedge\mathbf{b})\wedge\mathbf{c}
\end{equation}
\begin{equation}
    \mathbf{a}(\mathbf{b}\cdot\mathbf{c})+\mathbf{a}\cdot(\mathbf{b}\wedge\mathbf{c})=(\mathbf{a}\cdot\mathbf{b})\mathbf{c}+(\mathbf{a}\wedge\mathbf{b})\cdot\mathbf{c}
\end{equation}
\[
    \mathbf{a}(\mathbf{B}_r+\mathbf{C}_r)=\mathbf{a}\mathbf{B}_r+\mathbf{a}\mathbf{C}_r
\]
\begin{equation}\tag{2.1.9a}
    \mathbf{a}\cdot(\mathbf{B}_r+\mathbf{C}_r)=\mathbf{a}\cdot\mathbf{B}_r+\mathbf{a}\cdot\mathbf{C}_r
\end{equation}
\begin{equation}\tag{2.1.9b}
    \mathbf{a}\wedge(\mathbf{B}_r+\mathbf{C}_r)=\mathbf{a}\wedge\mathbf{B}_r+\mathbf{a}\wedge\mathbf{C}_r
\end{equation}
\setcounter{equation}{9}%
\begin{equation}
    A=\sum_r\langle A\rangle_r=\langle A\rangle_0+\langle A\rangle_1+\langle A\rangle_2+\langle A\rangle_3
\end{equation}
\begin{equation}\tag{2.1.11a}
    A=\langle A\rangle_++\langle A\rangle_-
\end{equation}
\begin{equation}\tag{2.1.11b}
    \langle A\rangle_+=\langle A\rangle_0+\langle A\rangle_2
\end{equation}
\begin{equation}\tag{2.1.11c}
    \langle A\rangle_-=\langle A\rangle_1+\langle A\rangle_3
\end{equation}
Klammerung
\begin{equation}\tag{2.1.12a}
    (A \wedge B)C=A \wedge BC\neq A\wedge(BC)
\end{equation}
\begin{equation}\tag{2.1.12b}
    (A \cdot B)C=A \cdot BC\neq A\cdot(BC)
\end{equation}
\setcounter{equation}{12}%
\begin{equation}
    A \cdot (B\wedge C)=A \cdot B\wedge C\neq (A\cdot B)\wedge C
\end{equation}
\begin{equation}
    \mathbf{a}\cdot(\mathbf{b}\wedge\mathbf{c})=(\mathbf{a}\cdot\mathbf{b})\mathbf{c}-(\mathbf{a}\cdot\mathbf{c})\mathbf{b}
\end{equation}
\end{document}
