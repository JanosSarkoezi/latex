\documentclass[11pt, a4paper]{article}
\usepackage{amsmath}
\usepackage[utf8]{inputenc}
\usepackage[ngerman]{babel}

\title{Trivial}
\author{Janos Sarközi}

\begin{document}

\maketitle
\newpage
\tableofcontents
\newpage

Wir beschäftigen uns mit trivialen Aussagen der Mathematik, die in den meisten
Büchern durch den Satz "Das ist trivial" behandelt wird, also garnicht. In
machen Fällen hat man das gefühl, der Author wollte den Teil des Beweises
nicht ausführlich diskutieren. Was für den einen trivial ist, muss für andere
nicht so sein.

Ich nehme Sätze aus verschiedenen gebieten der Mathematik und versuche die
trivialen Aussagen zu beleuchten.

\newpage
\section{Mengenlehre} Die Mengenlehre ist die Grundlage von allem und sollte
gut behandelt werden.  Hier bietet sich \cite{Ebbinghaus2003} an. Sehr
schönes Buch. Für das Verständnis der Mengenlehre beginnen wir mit einem
Zitat aus diesem Buch.
\begin{quote}
    Als Symbol für die Gleichheitsbeziegung zwischen Mengen werden wir Zeichen
    $=$, als Symbol für die Elementbezieung das Zeichen $\in$. Als
    \textit{Variablen für Mengen} dienen $v_0, v_1, v_2, ...$; um einer teils
    bequemeren, teils suggestiveren Schreibweise willen weichen wir jedoch
    häufig von dieser Konvention ab und verweden als Variablen beliebige
    kleine oder große Buchstaben vom Anfang oder Ende des Alphabets, nach Bedarf
    mit zusätzlichen Indizes oder anderen Unterscheidungszeichen.

    Die einfachsten \textit{Ausdrücke} unserer Sprache sind Feststellungen
    über die Gleichheit von Mengen und die Elementbezieung zwischen Mengen und
    habendie Gestalt $Variable = Variable$ bzw. $Variable \in Variable$, also
    z. B.
    \[
        v_0 = v_{17}, v_3 \in v_2, v_7 \in v_7
    \]%
\end{quote}
In diesem Buch werden die Junktoren $\lnot, \land, \lor, \rightarrow,
\leftrightarrow$ in der Reienfolge für \textit{nicht, und, oder, wenn ...
dann, genau dann ... wenn} benutzt. Dann gibt es noch die Quantoren $\forall,
\exists$, die stehen für die Beschrebung eines Ausdrucks \textit{für alle, es
gibt ein} in der Reienfolge.
\newline

In der Mengenlehre wird also mit Ausdrücken gearbeitet. Ein Ausdruck aus dem
Buch ist zum Beispiel
\[
    \varphi := \exists u\forall x(x\in y \rightarrow \exists y(y\in z))
\]
Mit diesem Ausdruck werden die freien Variablen behandelt, sie spielen die
Rolle von Parametern. Es kommt raus, dass der Ausdruck $\varphi$
deutlichkeitshalber auch $\varphi(y, z)$ geschrieben werden kann, sodaß wir
\[
    \varphi(y, z) := \exists u\forall x(x\in y \rightarrow \exists y(y\in z))
\]
schreiben können.
\subsection{Die leere Menge $\emptyset\subseteq X$}
Fangen wir mit dem trivialen Ausdruck \[ \emptyset \subseteq X \] an. Was
bedeuten die Symbole $\emptyset$, $\subseteq$ und $X$? Um diese Symbole zu
verstehen machen wir eine kleine Reise im Buch von \cite{Ebbinghaus2003}. In
diesem Buch wird $\subseteq$ durch ein Prädikat eingeführt. Am besten zitiere
ich direkt den Author des Buches:
\begin{quote}
    \textbf{Prädikate.} Der Ausdruck
    \[
        \varphi_T(x,y):=\forall z(z\in x \rightarrow z\in y)
    \]
    besagt, das jedes Element von $x$ auch ein Element von $y$ ist, d. h. dass
    ein $x$ eine \textit{Teilmenge} von y ist. Er beschreibt damit eine
    (zweistellige) \textit{Beziehung} zwischen Mengen, die
    \textit{Teimmengenbeziehung}. Eine Beziehung, die durch einen Ausdruck
    beschrieben werden kann, nennen wir ein \textit{Prädikat}. Die
    Teimmengenbeziehung ist also ein (zweistelliges) Prädikat.

    Für Prädikate führt man häufig neue Symbole, sog.
    \textit{Prädikats-symbole} ein. Für die Teimmengenbeziehung verwendet man
    ge-wöhnlich das (zweistellige) Prädikatssymbol $\subseteq$. Man definiert
    hierzu
    \[
        x \subseteq y :\leftrightarrow \varphi_T(x,y).
    \]
    Die Einführung von $\subseteq$ bedeutet eine \textit{Erweiterung der
    mengentheoretischen Sprache}, da z. B. Ausdrücke wie $x \subseteq x,
    x\subseteq y$ als neue atomare Ausdrücke hinzutreten.
\end{quote}
Somit ist zum Beispiel $\varphi(y) := \exists z(z\in y)$ ein einstelliges
Prädikat und die (zweistellige) Elementbezieung $x\in y$ ist ein
(zweistelliges) Prädikat. Der nächste Schritt ist die Einführung von
Operationen. Bei der Einführung von Operationen, wird mit der leeren Menge als
Beispiel argumentiert. Dabei wird die leere Menge als ein Konstantensymbol
eingeführt.
\begin{quote}
    \textbf{Operationen.} Die Abbildung, die jeder Menge $x$ des Universums
    die Menge der Teilmengen von $x$, die sog. \textit{Potenzmenge} von $x$,
    zuordnet, wollen wir die \textit{Potenzmengenabbildung} nennen. die
    Potenzmengenabbildung können wir durch einen Ausdruck $\varphi_P(x,y)$
    beschreiben, der gerade besagt, daß $y$ die Potenzmenge von $x$ ist,
    nähmlich durch
    \[
        \varphi_P(x,y) := \forall z(z\in y\leftrightarrow\varphi_T(z, x)).
    \]
    Eine auf dem Universum definierte Abbildung, die, wie die
    Potenzmengenabbildung, durch einen Ausdruck, also durch ein Prä-dikat,
    beschrieben werden kann, nennen wir eine \textit{Operation}.

    Bei Operationen stellt sich nun, anders als bei Prädikaten, eine
    spezifische Schwierigkeit ein. Auf Grund Unserer inhaltlichen
    Mengenvorstellung ist die Potenzmengenabbildung \textit{wohldefiniert}, d.
    h. es gibt zu \textit{jedem x genau ein y mit} $\varphi_P(x,y)$. Wenn wir
    also axiomatisch vorgehen, stehen uns nur die Axiome als Information zu
    Verfügung und wir können die Potenzmengenabbildung erst dann als
    wohldefiniert ansehen, wenn sich aus den betreffenden Axiomen
    \textit{beweisen} läßt, das es zu jedem $x$ genau ein $y$ gibt mit
    $\varphi_P(x,y)$. Ob also $\varphi_P(x,y)$ eine Operation beschreibt,
    hängt von dem zugrunde gelegten Axiomensystem ab.

    Allgemein: Es sei $\varphi(x,y)$ ein Ausdruck. Dann stehe "$\varphi(x,y)$
    \textit{ist funktional}" für den Ausdruck $\forall x\exists^{=1}y\,
    \varphi(x,y)$. Ist $\varphi(x,y)$ funktional, (d.h. gilt dieser Ausdruck)
    so definiert eine Abbildung auf dem Universum, die jeder Menge $x$
    \textit{das} $y$ mit $\varphi(x,y)$ zuordnet, mithin einer Operation. Auf
    die Funktionalität können wir aber nur dann zurückgreifen, falls sie aus den
    zugrunde gelegten Axiomen \textit{beweisbar} ist.

    Für Operationen führt man häufig neue Symbole, sog.
    \textit{Operationssymbole} ein, so etwa für die Potenzmengenabbildung, das
    Symbol Pot ein. Wir setzen dazu fest:
    \[
        \textrm{Pot}(x) = y :\leftrightarrow\varphi_P(x,y).
    \]
    aber, um es noch einmal zu sagen, dies erst dann, wenn wir die
    Funktionalität von $\varphi_P(x,y)$ gesichert haben.
    
    ...

    Statt \textit{einstelliger} Operationen lassen sich völlig analog
    \textit{zweistellige, dreistellige, ...} Operationen einführen. Ein
    Sonderfall ist die Stellenzahl 0: Wenn ein Ausdruck $\varphi(y)$ in dem
    Sinne funktional ist, dass $\exists^{=1}\varphi(y)$ gilt, so beschreibt
    $\varphi(y)$ eindeutig eine Menge. Für diese Menge können wir ein
    \textit{Konstantensymbol}, etwa $c$, dessen Bedeutung durch die Definition
    \[
        c = y :\leftrightarrow \varphi(y)
    \]
    festlegen. Zum Beispiel können wir schon mit wenigen Axiomen zeigen, dass
    der Ausdruck $\varphi(y)=\neg\exists z~z\in y$ ("y ist leer") funktional
    ist, und dann mit
    \[
        \emptyset = y :\leftrightarrow \neg\exists z~z\in y
    \]
    das übliche Konstantensymbol für die leere Menge einführen.
\end{quote}
Nun ist es von vorteil, wenn noch ein paar Axiome der Mengenlehre betrachtet
weden, um die Leere Menge zu verstehen.

\begin{quote}
\textbf{Existenzaxiom (Ex):}
\textit{Ex gibt eine Menge.}
Also
\[
    \exists x~x=x
\]
\textbf{Extensionalitätsaxiom (Ext):}
\textit{Umfangsgleiche Mengen sind gleich.}
Also
\[
    \forall xy(\forall z(z\in x \leftrightarrow z\in y) \leftrightarrow x=y)
\]
\textbf{Ext} gebietet Mengen nur unter quantitaitiven Gesichtspunkten zu sehen.
In der von uns gewählten Form hat es eine Konsequenz, die für die Gestallt der
Mengenlehre wesentlich ist: Wir aben uns enschlossen die Urelemnete asl Mengen
aufzufassen, die dann selbstferständlich nicht selber Elemente enthalten sollen,
also \textit{leer} sind. Nach \textbf{Ext} sind all diese Mengen gleich, d. h.
\textbf{Ext} \textit{schließt die Existenz verschiedene Urelemente aus}.

\textbf{Schema der Aussonderungsaxiome (Aus):}
Das Schema ent-hält zu jedem Ausdruck der Gestallt $\varphi(z,
\overset{n}{x})$\footnote{Dabei soll $\varphi(z, \overset{n}{x})$ eine
Abkürzung für $\varphi(z,x_1, ..., x_n)$ sein.} aus der ursprügli-chen
mengentheoretischen Sprache das Axiom

\textit{Zu allen $x_1, ..., x_n$ und zu allen x gibt es ein y, dass genau
diejenigen Elemente z von x enthält, für die $\varphi(z, \overset{n}{x})$ gilt.} 

Also
\[
    \forall\overset{n}{x}\forall x\exists y(\forall z(z\in y \leftrightarrow z\in
    x\land \varphi(z, \overset{n}{x}))
\]
\end{quote}

\noindent
Im Buch wird noch darauf hingewiesen, dass die Menge $y$, die die Bedingung
\[
\forall z(z\in y \leftrightarrow z\in x\land \varphi(z, \overset{n}{x})
\]
\textbf{Ext}
erfüllt, eindeutig bestimmt ist. Damit wird eine $n+1$ stellige Operation auf
dem Universum beschrieben und kann hiermit vereinbart als
\[
    \{z\in x: \varphi(z, \overset{n}{x})\}
\]
abgekürzt geschrieben werden. Wir nähern uns der entgültigen Definition der
leeren Menge. Hierzu weiter im Buch

\begin{quote}
    \textbf{Die leere Menge.} Sei $x$ eine Menge (\textbf{Ex!}). Nach
    \textbf{Aus} existiert dann die Menge $\{z\in x: z\neq z\}$. Sie ist leer
    und nach \textbf{Ext} eindeutig bestimmt.
\end{quote}

Im Buch wird für solche Fälle noch eine Vereinbarung für eine verkürzte
Schreibweise getroffen: Wenn es eine Menge gibt, die genau aus den $z$ mit
$\varphi(z, \overset{n}{x})$ besteht, schreiben wir für diese Menge auch
\[
    \{z:\varphi(z,\overset{n}{x})\}
\]
Damit kann die leere Menge als
\[
    \emptyset:=\{z:z\neq z\}
\]
gesetzt werden.

Jetz haben wir alle Zutaten, um zu entscheiden zu können, ob der
Ausdruck
\[
    \emptyset \subseteq X
\]
wahr ist, oder nicht. Wir müssen nur alle Symbole eliminieren und erhalten so
\begin{equation}
\begin{split}
    \emptyset \subseteq X & \quad \leftrightarrow\quad \varphi_T(\emptyset, X) \\ 
     & \quad\leftrightarrow\quad\forall z(z\in \emptyset \rightarrow z\in X) \\
     & \quad\leftrightarrow\quad\forall z(z\in \emptyset \rightarrow z\in X) \\
     & \quad\leftrightarrow\quad\forall z(z\neq z \rightarrow z\in X) \\
     & \quad\leftrightarrow\quad\forall z(\lnot(z=z) \rightarrow z\in X) \\
     & \quad\leftrightarrow\quad\forall z(z=z \lor z\in X)
\end{split}
\end{equation}
Damit ist der Ausdruck $\emptyset\subseteq X$ war. Anders gesagt, die leere Menge
ist in jeder Menge enthalten.
\subsection{Die leere Menge $X\cap\emptyset = \emptyset$}
Als nächstes können wir den Ausdruck
\[
X\cap\emptyset = \emptyset
\]
anschauen. Dafür betrachten wir die Definition von
\[
    x\cap y := \{z\in x: z\in y\}
\]
dem Durchschnitt zweier Mengen und erhalten so
\begin{equation}
\begin{split}
X\cap\emptyset = \emptyset
& \quad\leftrightarrow\quad \forall z(z\in (X\cap\emptyset) \leftrightarrow
z\in\emptyset) \\
& \quad\leftrightarrow\quad\forall z(z\in X \land z\in\emptyset
\leftrightarrow z\in\emptyset) \\
& \quad\leftrightarrow\quad\forall z(z\in X \land z\neq z \leftrightarrow z
\neq z) \\
\end{split}
\end{equation}
Der Durchschnitt der leeren Menge mit einer beliebigen Menge ist die leere
Menge.
\newpage
\bibliographystyle{plainnat}
\bibliography{books}

\end{document}
