\documentclass[11pt, a4paper]{article}
\usepackage{amsmath}
\usepackage[utf8]{inputenc}
\usepackage[ngerman]{babel}
% \usepackage{indentfirst}

\title{Trivial}
\author{Janos Sarközi}

\begin{document}

\maketitle
\newpage
\tableofcontents
\newpage

Wir beschäftigen uns mit trivialen Aussagen der Mathematik, die in den meisten
Büchern durch den Satz "Das ist trivial" behandelt wird, also garnicht. In
machen Fällen hat man das gefühl, der Author wollte den Teil des Beweises
nicht ausführlich diskutieren. Was für den einen trivial ist, muss für andere
nicht so sein.

Ich nehme Sätze aus verschiedenen gebieten der Mathematik und versuche die
trivialen Aussagen zu beleuchten.

\newpage
\section{Mengenlehre} Die Mengenlehre ist die Grundlage von allen und sollte
gut behandelt werden.  Hier bietet sich \cite{Ebbinghaus2003} an. Sehr
schönes Buch. Für das Verständnis der Mengenlehre beginnen wir mit einem
Zitat aus diesem Buch.
\begin{quote}
    Als Symbol für die Gleichheitsbeziegung zwischen Mengen werden wir Zeichen
    $=$, als Symbol für die Elementbezieung das Zeichen $\in$. Als
    \textit{Variablen für Mengen} dienen $v_0, v_1, v_2, ...$; um einer teils
    bequemeren, teils suggestiveren Schreibweise willen weichen wir jedoch
    häufig von dieser Konvention ab und verweden als Variablen beliebige
    kleine oder große Buchstaben vom Anfang oder Ende des Alphabets, nach Bedarf
    mit zusätzlichen Indizes oder anderen Unterscheidungszeichen.

    Die einfachsten \textit{Ausdrücke} unserer Sprache sind Feststellungen
    über die Gleichheit von Mengen und die Elementbezieung zwischen Mengen und
    habendie Gestalt $Variable = Variable$ bzw. $Variable \in Variable$, also
    z. B.
    \[
        v_0 = v_{17}, v_3 \in v_2, v_7 \in v_7
    \]
\end{quote}
\noindent
In diesem Buch werde die Junktoren $\lnot, \land, \lor, \rightarrow,
\leftrightarrow$ in der Reienfolge für \textit{nicht, und, oder, wenn ...
dann, genau dann ... wenn} benutzt. Dann gibt es noch die Quantoren $\forall,
\exists$, die stehen für die Beschrebung eines Ausdrucks \textit{für alle, es
gibt ein} in der Reienfolge.
\newline

\noindent
In der Mengenlehre wird also mit Ausdrücken gearbetet. Ein Ausdruck aus dem
Buch ist zum Beispiel
\[
    \varphi := \exists u\forall x(x\in y \rightarrow \exists y(y\in z))
\]
Mit diesem Ausdruck werden die freien Variablen behandelt, sie spielen die
Rolle von Parametern. Es kommt raus. dass der Ausdruck $\varphi$
deutlichkeitshalber auch $\varphi(y, z)$ geschrieben werden kann, sodaß wir
sommit 
\[
    \varphi(y, z) := \exists u\forall x(x\in y \rightarrow \exists y(y\in z))
\]
schreiben können.
\newline

\noindent
Fangen wir mit dem trivialen Ausdruck \[ \emptyset \subseteq X \] an. Was
bedeuten die Symbole $\emptyset$, $\subseteq$ und $X$? Das Symbol $\emptyset$
ist die leere Menge. Um die leere Menge zu verstehen, machen wir ein Schritt
zurück und versuchen eine Menge zu beschreiben, die ein Element hat. Der
folgende Ausdruck beschreibt diesen Sachverhalt
\[
    \varphi(y) := \exists z(z\in y)
\]
Sie besagt in Worten: $y$ hat ein Element. Nun müssen wir diesen Satz nur noch
verneinen und sagen $y$ hat kein Element oder etwas mathematischer, es gibt kein
Element in $y$. In Symbolen also
\[
    \psi(y) :=\lnot\varphi(y) = \lnot\exists z(z\in y)
\]
Und dieses $y$ ist unsere leere Menge $\emptyset$. Nun zu dem Symbol
$\subseteq$. In dem Buch von \cite{Ebbinghaus2003} wird $\subseteq$ durch
ein Prädikat eingeführt und die leere Menge $\emptyset$ duch eine
Operation. Am besten zitiere ich direkt den Author des Buches
\begin{quote}
    \textbf{Prädikate.} Der Ausdruck
    \[
        \varphi_T(x,y):=\forall z(z\in x \rightarrow z\in y)
    \]
    besagt, das jedes Element von $x$ auch ein Element von $y$ ist, d. h. dass
    ein $x$ eine \textit{Teilmenge} von y ist. Er beschreibt damit eine
    (zweistellige) \textit{Beziehung} zwischen Mengen, die
    \textit{Teimmengenbeziehung}. Eine Beziehung, die durch einen Ausdruck
    beschrieben werden kann, nennen wir ein \textit{Prädikat}. Die
    Teimmengenbeziehung ist also ein (zweistelliges) Prädikat.

    Für Prädikate führt man häufig neue Symbole, sog.
    \textit{Prädikats-symbole} ein. Für die Teimmengenbeziehung verwendet man
    ge-wöhnlich das (zweistellige) Prädikatssymbol $\subseteq$. Man definiert
    hierzu
    \[
        x \subseteq y :\leftrightarrow \varphi_T(x,y).
    \]
    Die Einführung von $\subseteq$ bedeutet eine \textit{Erweiterung der
    mengentheoretischen Sprache}, da z. B. Ausdrücke wie $x \subseteq x,
    x\subseteq y$ als neue atomare Ausdrücke hinzutreten.
\end{quote}
Somit ist zum Beispiel $\varphi(y) := \exists z(z\in y)$ ein einstelliges
Prädikat und die (zweistellige) Elementbezieung $x\in y$ ist ein
(zweistelliges) Prädikat. Der nächste Schritt ist die Einführung von
Operationen.
\begin{quote}
    \textbf{Operationen.} Die Abbildung, die jeder Menge $x$ des Universums
    die Menge der Teilmengen von $x$, die sog. \textit{Potenzmenge} von $x$,
    zuordnet, wollen wir die \textit{Potenzmengenabbildung} nennen. die
    Potenzmengenabbildung können wir durch einen Ausdruck $\varphi_P(x,y)$
    beschreiben, der gerade besagt, daß $y$ die Potenzmenge von $x$ ist,
    nähmlich durch
    \[
        \varphi_P(x,y) := \forall z(z\in y\leftrightarrow\varphi_T(z, x)).
    \]
    Eine auf dem Universum definierte Abbildung, die, wie die
    Potenzmengenabbildung, durch einen Ausdruck, also durch ein Prä-dikat,
    beschrieben werden kann, nennen wir eine \textit{Operation}.

    Bei Operationen stellt sich nun, anders als bei Prädikaten, eine
    spezifische Schwierigkeit ein. Auf Grund Unserer inhaltlichen
    Mengenvorstellung ist die Potenzmengenabbildung \textit{wohldefiniert}, d.
    h. es gibt zu \textit{jedem x genau ein x mit} $\varphi_P(x,y)$. Wenn wir
    also axiomatisch vorgehen, stehen uns nur die Axiome als Information zu
    Verfügung und wir können die Potenzmengenabbildung erst dann als
    wohldefiniert ansehen, wenn sich aus den betreffenden Axiomen
    \textit{beweisen} läßt, das es zu jedem $x$ genau ein $y$ gibt mit
    $\varphi_P(x,y)$. Ob also $\varphi_P(x,y)$ eine Operation beschreibt,
    hängt von dem zugrunde gelegten Axiomensystem ab.

    Allgemein: Es sei $\varphi(x,y)$ ein Ausdruck. Dann stehe "$\varphi(x,y)$
    \textit{ist funktional}" für den Ausdruck $\forall x\exists^{=1}y\,
    \varphi(x,y)$.
    
    ...

    Für Operationen führt man häufig neue Symbole, sog.
    \textit{Operationssymbole} ein, so etwa für die Potenzmengenabbildung, das
    Symbol Pot ein. Wir setzen dazu fest:
    \[
        \textrm{Pot}(x) = y :\leftrightarrow\varphi_P(x,y).
    \]
    aber, um es noch einmal zu sagen, dies erst dann, wenn wir die
    Funktionalität von $\varphi_P(x,y)$ gesichert haben.
\end{quote}


Jetz haben wir alle Zutaten, um zu entscheiden zu können, ob der
Ausdruck
\[
    \emptyset \subseteq X
\]
wahr ist, oder nicht. Wir müssen nur alle einsetzen und erhalten
\begin{equation}
\begin{split}
    \emptyset \subseteq X & \quad gdw\quad \varphi_T(\emptyset, X) \\ 
     & \quad gdw\quad\forall z(z\in \emptyset \rightarrow z\in X)
\end{split}
\end{equation}

\newpage \bibliographystyle{plainnat} \bibliography{books} \end{document}
