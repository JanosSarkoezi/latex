\documentclass[leqno]{article}           % leqno für die Linksnummerierung von Formeln
\usepackage[utf8]{inputenc}              % Benutzug von UTF-8 Textdateien
\usepackage[T1]{fontenc}                 % Umlaute und co
\usepackage[ngerman]{babel}              % Deutsche Trennung von Wörtern
\usepackage{amsmath}                     % Benutzung von flalign
\usepackage{amssymb}                     % Mathematische Symbole (Natürliche Zahlen ...)
\usepackage[showframe]{geometry}         % Anzeigen des Seitenaufbaus
\usepackage{stmaryrd}                    % Das Blitzsybol für Widersprüche

\usepackage{tikz}
\begin{document}

\tikzstyle{a}=[draw=green,fill=green!30,
    shape=rectangle,
    rounded corners,
    draw, align=center,
    top color=white,
    bottom color=blue!20]
\tikzstyle{b}=[draw=red,fill=red!30,
    shape=rectangle,
    rounded corners,
    draw, align=center,
    top color=white,
    bottom color=blue!20]
\tikzstyle{c}=[draw=blue,fill=green!30,
    shape=rectangle,
    rounded corners,
    draw, align=center,
    top color=white,
    bottom color=blue!20]

\begin{figure}[hbp]
  \begin{minipage}[t][1cm][t]{.1\linewidth}
    \begin{tikzpicture}
      \node[a]               (01) {$\neg\neg\varphi$};
      \node[a, below of=01]  (02) {$\varphi$};
    
      \draw (01) -- (02);
    \end{tikzpicture}
  \end{minipage}
  \begin{minipage}[t][3cm][t]{.1\linewidth}
    \begin{tikzpicture}
      \node[a]              (01) {$\varphi\land\psi$};
      \node[a, below of=01] (02) {$\varphi$};
      \node[a, below of=02] (03) {$\psi$};
    
      \draw (01) -- (02) -- (03);
    \end{tikzpicture}
  \end{minipage}
  \begin{minipage}[t][3cm][t]{.24\linewidth}
    \begin{tikzpicture}
      \node[a]                            (01) {$\varphi\lor\psi$};
      \node[a, below of=01, xshift= 10mm] (02) {$\psi$};
      \node[a, below of=01, xshift=-10mm] (03) {$\varphi$};
    
      \draw (01.south) -- ++(0mm, -2mm) -- ++( 10mm, 0mm) -- (02.north); 
      \draw (01.south) -- ++(0mm, -2mm) -- ++(-10mm, 0mm) -- (03.north); 
    \end{tikzpicture}
  \end{minipage}
  \begin{minipage}[t][3cm][t]{.25\linewidth}
    \begin{tikzpicture}
      \node[a]                            (01) {$\varphi\rightarrow\psi$};
      \node[a, below of=01, xshift= 10mm] (02) {$\psi$};
      \node[a, below of=01, xshift=-10mm] (03) {$\neg\varphi$};
    
      \draw (01.south) -- ++(0mm, -2mm) -- ++( 10mm, 0mm) -- (02.north); 
      \draw (01.south) -- ++(0mm, -2mm) -- ++(-10mm, 0mm) -- (03.north); 
    \end{tikzpicture}
  \end{minipage}
  \begin{minipage}[t][3cm][t]{.25\linewidth}
    \begin{tikzpicture}
      \node[a]                            (01) {$\varphi\leftrightarrow\psi$};
      \node[a, below of=01, xshift= 10mm] (02) {$\neg\varphi$};
      \node[a, below of=02] (04) {$\neg\psi$};
      \node[a, below of=01, xshift=-10mm] (03) {$\varphi$};
      \node[a, below of=03] (05) {$\psi$};
    
      \draw (01.south) -- ++(0mm, -2mm) -- ++( 10mm, 0mm) -- (02.north); 
      \draw (01.south) -- ++(0mm, -2mm) -- ++(-10mm, 0mm) -- (03.north); 
      \draw (02) -- (04);
      \draw (03) -- (05);
    \end{tikzpicture}
  \end{minipage}
\end{figure}

\newpage
Der Beweis von
\begin{align*}
    \forall x\forall y\varphi(x,y) \models\forall y\forall x\varphi(x, y)
\end{align*}
als Warheitsbaum.

\begin{figure}[h]
\centering
\begin{tikzpicture}
    \node (01) [a] {$\forall x\forall y\varphi(x, y)$ (01)};
    \node (02) [b, below of=01] {$\neg\forall y\forall x\varphi(x, y)$ (02)};
    \node (03) [b, below of=02] {$\exists y\exists x\neg\varphi(x, y)$ (03)};
    \node (04) [b, below of=03] {$\exists x\neg\varphi(x, a)$ (04)};
    \node (05) [b, below of=04] {$\neg\varphi(b, a)$ (05)};
    \node (06) [a, below of=05] {$\forall y\varphi(b, y)$ (06)};
    \node (07) [a, below of=06] {$\varphi(b, a)$ (07)};

    \node (08) [below of=07, yshift=3mm] {$(07)\lightning(05)$};

    \draw (01) -- (02) -- (03) -- (04) -- (05) -- (06) -- (07);
\end{tikzpicture}
\caption{Die eine Richtung der Äquivalenz}
\label{fig:allAllChanged01}
\end{figure}

\newpage
Nun die andere Richtung der Aussage
\begin{align*}
    \forall y\forall x\varphi(x,y) \models\forall x\forall y\varphi(x, y)
\end{align*}
als Warheitsbaum.

\begin{figure}[h]
\centering
\begin{tikzpicture}
    \node (01) [a] {$\forall y\forall x\varphi(x, y)$ (01)};
    \node (02) [b, below of=01] {$\neg\forall x\forall y\varphi(x, y)$ (02)};
    \node (03) [b, below of=02] {$\exists x\exists y\neg\varphi(x, y)$ (03)};
    \node (04) [b, below of=03] {$\exists y\neg\varphi(a, y)$ (04)};
    \node (05) [b, below of=04] {$\neg\varphi(a, b)$ (05)};
    \node (06) [a, below of=05] {$\forall y\varphi(a, y)$ (06)};
    \node (07) [a, below of=06] {$\varphi(a, b)$ (07)};

    \node (08) [below of=07, yshift=3mm] {$(07)\lightning(05)$};

    \draw (01) -- (02) -- (03) -- (04) -- (05) -- (06) -- (07);
\end{tikzpicture}
\caption{Die eine Richtung der Äquivalenz}
\label{fig:allAllChanged02}
\end{figure}

\newpage
Der Beweis von
\begin{align*}
    \exists x\forall y(\varphi(y)\leftrightarrow x=y)) \models\exists x(\varphi(x)\wedge\forall y(\varphi(y)\to x=y))
\end{align*}
als Warheitsbaum.

\begin{figure}[h]
\centering
\begin{tikzpicture}
    \node (01) [a] {$\exists x\forall y(\varphi(y)\leftrightarrow x=y))$ (01)};
    \node (02) [b, below of=01] {$\neg\exists x(\varphi(x)\wedge\forall y(\varphi(y)\to x=y))$ (02)};
    \node (03) [b, below of=02] {$\forall x\neg(\varphi(x)\wedge\forall y(\varphi(y)\to x=y))$ (03)};
    \node (04) [a, below of=03] {$\forall y(\varphi(y)\leftrightarrow a=y)$ (04)};
    \node (05) [b, below of=04] {$\neg(\varphi(a)\wedge\forall y(\varphi(y)\to a=y)$ (05)};
    \node (06) [b, below of=05, xshift=-35mm] {$\neg\varphi(a)$ (06)};
    \node (07) [b, below of=05, xshift= 35mm] {$\neg\forall y(\varphi(y)\to a=y)$ (07)};
    \node (08) [a, below of=06] {$\varphi(a)\leftrightarrow a=a$ (08)};
    \node (09) [a, below of=08, xshift=-15mm] {$\varphi(a)$ (09)};
    \node (10) [a, below of=09] {$a=a$ (10)};
    \node (11) [a, below of=08, xshift= 15mm] {$\neg\varphi(a)$ (11)};
    \node (12) [a, below of=11] {$a\neq a$ (12)};
    \node (13) [c, below of=12] {$a=a$ (13)};
    \node (14) [b, below of=07] {$\exists y\neg(\varphi(y)\to a=y)$ (14)};
    \node (15) [b, below of=14] {$\neg(\varphi(b)\to a=b)$ (15)};
    \node (16) [b, below of=15] {$\varphi(b)$ (16)};
    \node (17) [b, below of=16] {$a\neq b$ (17)};
    \node (18) [a, below of=17] {$\varphi(b)\leftrightarrow a=b$ (18)};
    \node (19) [a, below of=18, xshift=-15mm] {$\varphi(b)$ (19)};
    \node (20) [a, below of=19] {$a=b$ (20)};
    \node (21) [a, below of=18, xshift= 15mm] {$\neg\varphi(b)$ (21)};
    \node (22) [a, below of=21] {$a\neq b$ (22)};

    \node (23) [below of=22, yshift=3mm] {$(21)\lightning(16)$};
    \node (24) [below of=20, yshift=3mm] {$(20)\lightning(17)$};
    \node (25) [below of=13, yshift=3mm] {$(13)\lightning(12)$};
    \node (26) [below of=10, yshift=3mm] {$(09)\lightning(06)$};

    \draw (01) -- (02) -- (03) -- (04) -- (05);
    \draw (05.south) -- ++(0mm, -2mm) -- ++(-35mm, 0mm) -- (06.north); 
    \draw (05.south) -- ++(0mm, -2mm) -- ++( 35mm, 0mm) -- (07.north); 
    \draw (06) -- (08);
    \draw (08.south) -- ++(0mm, -2mm) -- ++(-15mm, 0mm) -- (09.north); 
    \draw (08.south) -- ++(0mm, -2mm) -- ++( 15mm, 0mm) -- (11.north); 
    \draw (09) -- (10); 
    \draw (11) -- (12) -- (13);
    \draw (07) -- (14) -- (15) -- (16) -- (17) -- (18);
    \draw (18.south) -- ++(0mm, -2mm) -- ++(-15mm, 0mm) -- (19.north); 
    \draw (18.south) -- ++(0mm, -2mm) -- ++( 15mm, 0mm) -- (21.north); 
    \draw (19) -- (20); 
    \draw (21) -- (22); 

    % \draw (01.east) .. controls +(350:2cm) and +(10:2cm) .. node[right, swap] {$\exists x \to a$} (04.east);
    % \draw[bend right]    (02.west) to node[auto, swap] {$\neg$}            (03.west);
    % \draw[bend right]    (03.west) to node[auto, swap] {$\forall x \to a$} (05.west);
    % \draw[bend left=70]  (04.east) to node[auto]       {$\forall y \to a$} (08.east);

\end{tikzpicture}
\caption{Die eine Richtung der Äquivalenz}
\label{fig:onlyOneExists01}
\end{figure}

\newpage
Nun die andere Richtung der Aussage
\begin{align*}
    \exists x(\varphi(x)\wedge\forall y(\varphi(y)\to x=y)) \models \exists x\forall y(\varphi(y)\leftrightarrow x=y))
\end{align*}
als Warheitsbaum.

\begin{figure}[h]
\centering
\begin{tikzpicture}
    \node (01) [a] {$\exists x(\varphi(x)\wedge\forall y(\varphi(y)\to x=y))$ (01)};
    \node (02) [b, below of=01] {$\neg\exists x\forall y(\varphi(y)\leftrightarrow x=y))$ (02)};
    \node (03) [b, below of=02] {$\forall x\exists y\neg(\varphi(y)\leftrightarrow x=y))$ (03)};
    \node (04) [a, below of=03] {$\varphi(a)\wedge\forall y(\varphi(y)\to a=y)$ (04)};
    \node (05) [b, below of=04] {$\exists y\neg(\varphi(y)\leftrightarrow a=y))$ (05)};
    \node (06) [b, below of=05] {$\neg(\varphi(b)\leftrightarrow a=b))$ (06)};
    \node (07) [a, below of=06] {$\varphi(a)$ (07)};
    \node (08) [a, below of=07] {$\forall y(\varphi(y)\to a=y)$ (08)};
    \node (09) [a, below of=08] {$\varphi(b)\to a=b$ (09)};
    \node (10) [a, below of=09, xshift=-35mm] {$\neg\varphi(b)$ (10)};
    \node (11) [a, below of=09, xshift= 35mm] {$a=b$ (11)};
    \node (12) [b, below of=11, xshift=-15mm] {$\varphi(b)$ (12)};
    \node (13) [b, below of=11, xshift= 15mm] {$\neg\varphi(b)$ (13)};
    \node (14) [b, below of=12] {$a\neq b$ (14)};
    \node (15) [b, below of=13] {$a=b$ (15)};
    \node (16) [b, below of=10, xshift=-15mm] {$\varphi(b)$ (16)};
    \node (17) [b, below of=10, xshift= 15mm] {$\neg\varphi(b)$ (17)};
    \node (18) [b, below of=16] {$a\neq b$ (18)};
    \node (19) [b, below of=17] {$a=b$ (19)};
    \node (20) [c, below of=15] {$\neg\varphi(a)$ (20)};
    \node (21) [c, below of=19] {$\neg\varphi(a)$ (21)};

    \node (22) [below of=18, yshift=3mm] {$(16)\lightning(10)$};
    \node (23) [below of=21, yshift=3mm] {$(21)\lightning(07)$};
    \node (24) [below of=14, yshift=3mm] {$(14)\lightning(11)$};
    \node (25) [below of=20, yshift=3mm] {$(20)\lightning(07)$};

    \draw (01) -- (02) -- (03) -- (04) -- (05) -- (06) -- (07) -- (08) -- (09);
    \draw (09.south) -- ++(0mm, -2mm) -- ++( 35mm, 0mm) -- (11.north); 
    \draw (09.south) -- ++(0mm, -2mm) -- ++(-35mm, 0mm) -- (10.north); 
    \draw (11.south) -- ++(0mm, -2mm) -- ++( 15mm, 0mm) -- (13.north); 
    \draw (11.south) -- ++(0mm, -2mm) -- ++(-15mm, 0mm) -- (12.north); 
    \draw (10.south) -- ++(0mm, -2mm) -- ++( 15mm, 0mm) -- (17.north); 
    \draw (10.south) -- ++(0mm, -2mm) -- ++(-15mm, 0mm) -- (16.north); 
    \draw (13) -- (15) -- (20);
    \draw (17) -- (19) -- (21);
    \draw (16) -- (18);
    \draw (12) -- (14);

\end{tikzpicture}
\caption{Die andere Richtung der Äquivalenz}
\label{fig:onlyOneExists02}
\end{figure}

\newpage
Der Versuch eines Beweises von
\begin{align*}
    \forall x\exists y\varphi(x,y) \models\exists y\forall x\varphi(x, y)
\end{align*}
als Warheitsbaum, wird scheitern.

\begin{figure}[h]
\centering
\begin{tikzpicture}
    \node (01) [a] {$\forall x\exists y\varphi(x, y)$ (01)};
    \node (02) [b, below of=01] {$\neg\exists y\forall x\varphi(x, y)$ (02)};
    \node (03) [b, below of=02] {$\forall y\exists x\neg\varphi(x, y)$ (03)};
    \node (04) [b, below of=03] {$\exists x\neg\varphi(x, a)$ (04)};
    \node (05) [b, below of=04] {$\neg\varphi(b, a)$ (05)};
    \node (06) [a, below of=05] {$\exists y\varphi(b, y)$ (06)};
    \node (07) [a, below of=06] {$\varphi(b, c)$ (07)};
    \node (08) [b, below of=07] {$\exists x\neg\varphi(x, c)$ (08)};
    \node (09) [b, below of=08] {$\neg\varphi(d, c)$ (09)};
    \node (10) [a, below of=09] {$\exists y\varphi(d, y)$ (10)};
    \node (11) [a, below of=10] {$\varphi(d, e)$ (11)};

    % \node (08) [below of=07, yshift=3mm] {$(07)\lightning(05)$};

    \draw (01) -- (02) -- (03) -- (04) -- (05) -- (06) -- (07) -- (08) -- (09);
    \draw (09) -- (10) -- (11);
\end{tikzpicture}
\caption{Der Versuch eines Beweises}
\label{fig:allExistsChange01}
\end{figure}
Bei den Existenzaussagen müssen immer neue Individualkonstanten eingeführt werden, die
noch nicht im Baum weiter oben schon vorhanen waren, sodass kein Widerspruch hergeleitet
werden kann. Die Interpretation
\begin{align*}
    D&=\mathbb{N} \\
    \varphi(x, y)&= x < y
\end{align*}
führt zu dem Satz: 
\begin{align*}
    \forall x\exists y(x < y) \models\exists y\forall x(x < y)\quad{x,y\in\mathbb{N}}
\end{align*}

\newpage
Der Beweises von
\begin{align*}
    \exists x\forall y\varphi(x,y) \models\forall y\exists x\varphi(x, y)
\end{align*}
als Warheitsbaum.

\begin{figure}[h]
\centering
\begin{tikzpicture}
    \node (01) [a] {$\exists x\forall y\varphi(x, y)$ (01)};
    \node (02) [b, below of=01] {$\neg\forall y\exists x\varphi(x, y)$ (02)};
    \node (03) [b, below of=02] {$\exists y\forall x\neg\varphi(x, y)$ (03)};
    \node (04) [b, below of=03] {$\forall x\neg\varphi(x, a)$ (04)};
    \node (05) [a, below of=04] {$\forall y\varphi(b, y)$ (05)};
    \node (06) [b, below of=05] {$\neg\varphi(b, a)$ (06)};
    \node (07) [a, below of=06] {$\varphi(b, a)$ (07)};

    \node (08) [below of=07, yshift=3mm] {$(07)\lightning(06)$};

    \draw (01) -- (02) -- (03) -- (04) -- (05) -- (06) -- (07);
\end{tikzpicture}
\caption{Der Versuch eines Beweises}
\label{fig:existsAllChange01}
\end{figure}

\newpage
Der Satz
\begin{align*}
    \exists x\in\mathbb{N}\:\forall y\in\mathbb{N}\:(x < y) \models
    \forall y\in\mathbb{N}\:\exists x\in\mathbb{N}\:(x < y)
\end{align*}
ist nach der Mengenlehre identisch mit dem Satz
\begin{align*}
    \exists x(x\in\mathbb{N}\wedge\forall y(y\in\mathbb{N}\rightarrow(x < y)) \models
    \forall y(y\in\mathbb{N}\rightarrow\exists x(x\in\mathbb{N}\wedge(x < y))
\end{align*}

\begin{figure}[h]
\centering
\begin{tikzpicture}
    \node (01) [a] {%
        $\exists x(x\in\mathbb{N}\wedge\forall y(y\in\mathbb{N}\rightarrow(x < y))$ (01)%
    };
    \node (02) [b, below of=01] {
        $\neg\forall y(y\in\mathbb{N}\rightarrow\exists x(x\in\mathbb{N}\wedge(x < y))$ (02)%
    };
    \node (03) [b, below of=02] {
        $\exists y\neg(y\in\mathbb{N}\rightarrow\exists x(x\in\mathbb{N}\wedge(x < y))$ (03)%
    };
    \node (04) [b, below of=03] {
        $\neg(a\in\mathbb{N}\rightarrow\exists x(x\in\mathbb{N}\wedge(x < a))$ (04)%
    };
    \node (05) [b, below of=04] {$a\in\mathbb{N}$ (05)};
    \node (06) [b, below of=05] {$\neg\exists x(x\in\mathbb{N}\wedge(x < a))$ (06)};
    \node (07) [b, below of=06] {$\forall x\neg(x\in\mathbb{N}\wedge(x < a))$ (07)};
    \node (08) [a, below of=07] {%
        $b\in\mathbb{N}\wedge\forall y(y\in\mathbb{N}\rightarrow(b < y))$ (08)%
    };
    \node (09) [a, below of=08] {$b\in\mathbb{N}$ (09)};
    \node (10) [a, below of=09] {$\forall y(y\in\mathbb{N}\rightarrow(b < y))$ (10)};
    \node (11) [a, below of=10] {$a\in\mathbb{N}\rightarrow(b < a)$ (11)};
    \node (12) [a, below of=11, xshift=-35mm] {$a\not\in\mathbb{N}$ (12)};
    \node (13) [a, below of=11, xshift=35mm] {$b < a$ (13)};
    \node (14) [b, below of=13] {$\neg(b\in\mathbb{N}\wedge(b < a))$ (14)};
    \node (15) [b, below of=14, xshift=-15mm] {$b\not\in\mathbb{N}$ (15)};
    \node (16) [b, below of=14, xshift= 15mm] {$b \not< a$ (16)};

    \node (17) [below of=12, yshift=3mm] {$(12)\lightning(05)$};
    \node (18) [below of=15, yshift=3mm] {$(15)\lightning(09)$};
    \node (19) [below of=16, yshift=3mm] {$(16)\lightning(13)$};

    \draw (01) -- (02) -- (03) -- (04) -- (05) -- (06) -- (07) -- (08) -- (09);
    \draw (09) -- (10) -- (11);
    \draw (11.south) -- ++(0mm, -2mm) -- ++(-35mm, 0mm) -- (12.north); 
    \draw (11.south) -- ++(0mm, -2mm) -- ++( 35mm, 0mm) -- (13.north); 
    \draw (13) -- (14);
    \draw (14.south) -- ++(0mm, -2mm) -- ++(-15mm, 0mm) -- (15.north); 
    \draw (14.south) -- ++(0mm, -2mm) -- ++( 15mm, 0mm) -- (16.north); 
\end{tikzpicture}
\caption{Der Beweises über die Menge der natürlichen Zahlen}
\label{fig:exapleOfExistsAll}
\end{figure}

% vim: set filetype=tex


\end{document}
