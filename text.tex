\documentclass[10pt,twocolumn]{article}

\usepackage{xskak}

\begin{document}

\begin{center}
    \textbf{Botvinnik -- Smyslov} \\
    \textit{World Championship Match (17)} \\
    \textit{Moscow 1957} \\
    Gr\"{u}nfeld Defence
\end{center}

\newchessgame
\xskakset{style=test}

\mainline[level=1]{1. Nf3 Nf6 2. g3 g6 3. c4 c6 4. Bg2 Bg7 5. Nc3 O-O 6. d4 d5 7. cxd5
cxd5 8. Ne5 b6}

A satisfactory way of developing.  After {\variation[invar]{8... Nc6 9. Nxc6
bxc6}} Black would be saddled with a weak pawn at c6.

\mainline[outvar, outvar]{9. Bg5 Bb7 10. Bxf6 Bxf6 11. O-O}

In the variation {\variation[invar]{11. e4 dxe4 12. Nxe4 Bxe5 13. dxe5 Nd7
14. f4 Nc5}} Black has a good game.

\mainline{11... e6 12. f4}

If {\variation[invar]{12. e4 Nc6 13. exd5} (\variation[invar]{13. Nxc6 Bxc6
14. exd5 exd5}) \variation[outvar]{13... Nxe5 14. dxe5 Bxe5 15. dxe6 Bxg2 16.
exf7+ Rxf7 17. Kxg2 Bxc3 18. bxc3 Qxd1 19. Rfxd1 Rc8 20. Rd3 Rc4}}, and Black
can successfully battle for a draw in the rook ending.

\end{document}
